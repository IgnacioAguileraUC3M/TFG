En este apartado se presentan las conclusiones generales del trabajo, donde se recopilan todas lecciones aprendidas, al igual que se analizan los objetivos cumplidos. Adicionalmente se exponen las dificultades encontradas en el desarrollo y las posibles lineas de trabajo futuro.

\section{Conclusiones generales}
Como se estableció en la introducción, el objetivo principal de este trabajo era el poder clasificar de manera efectiva textos acorde a los \gls{ODSa} relacionados con el mismo. Después de un largo desarrollo y una exhaustiva investigación se puede afirmar que este ha sido cumplido. 

El desarrollo de una serie de modelos de aprendizaje profundo resulta un aprendizaje importante, más aún si se tiene en cuenta que se han seguido diversas arquitecturas y estructuras de datos por lo que, de manera adicional, se ha ayudado a entender mejor el impacto y efecto que estos aspectos tienen en el rendimiento final del modelo. Parte esencial esta evaluación es la exhaustiva validación que se ha llevado a cabo, parte esencial de la cual ha sido el uso de herramientas de análisis de datos que ayuden a entender de manera más profunda la estructura de estos demostrando al importancia de realizar un análisis de este tipo antes de analizar de manera final los resultados y no saltar a conclusiones de manera acelerada. 

El recopilado de datos de internet ha demostrados ser una herramienta extremadamente útil, abriendo un mundo de posibilidades infinito, proporcionando una cantidad inmensa de datos. Destacar también la importancia de realizar una extracción legitima de estos datos, siempre en concordancia con lo que dictamina el marco regulador. 

La investigación llevada a cabo al final, aunque menor en magnitud, representa un hito importante, demostrando la capacidad e importancia del modelos como los desarrollados en este proyecto a la hora de tratar con cantidades grandes y heterogéneas de datos. Y demostrando la posibilidad de adaptar estos modelos a temas con  taxonomías extensas y complicadas y obteniendo aún así, unos resultados que, teniendo en cuenta los resultados de las validaciones,  resultan fiables.

\section{Dificultades y limitaciones}
El desarrollo de modelos de aprendizaje profundo está plagado de limitaciones, en primer lugar se encuentran las limitaciones de hardware ya que, el uso de modelos grandes resulta una tarea pesada que no puede ser llevada a cabo por cualquier ordenador, el adaptar los modelos a estas limitaciones es siempre una tarea difícil y más aún cuando no se cuenta con un hardware puntero. En segundo lugar se encuentran las limitaciones de software ya que los entornos de desarrollo utilizados son altamente dependientes en la versión usada, si se implementa alguna característica relativamente novedosa esto podría dar lugar a incompatibilidades entre funcionalidades y módulos diferentes. 

Otra limitación referente a la creación de modelos de aprendizaje profundo, y aplicable a cualquier tarea de aprendizaje automático, es la recolección de los datos, esto puede ser una tarea difícil si no se cuenta con una base de datos extensa inicial. El recopilado de datos es una buena solución pero tiene que llevarse a cabo con cautela debido a al alta relación entre los datos usados y los resultados obtenidos. 

\section{Líneas de trabajo futuro}
Este trabajo se deja diversas funcionalidades, características e investigaciones en el tintero, abriendo un amplio abanico de lineas futuras de trabajo. Como linea principal se encuentra el adaptar la clasificación de únicamente textos en inglés a textos en otros idiomas como puede ser el español,facilitando así el monitoreo de diferentes entornos e instituciones. 

De manera adicional se podría hacer uso de modelos como los desarrollados en este trabajo para ampliar el estudio de  las relaciones entre los \gls{ODSa}, pudiendo por ejemplo identificar la frecuencia en la que los diferentes objetivos aparecen juntos, identificando, de esta manera, como se relacionan los objetivos entre y las influencias que unos tienen sobre otros.

Otros ámbitos en los que se podría desarrollar un estudio como el desarrollado en este podrían ser el conjunto de regulaciones ambientales y de sostenibilidad, en el caso de España estas regulaciones son impuestas a nivel europeo por lo que un estudio sobre estas podría resultar útil. Pudiendo así compara los esfuerzos científicos con los legislativos, identificando así diferencias o coincidencias de intereses. 