Como parte esencial de cualquier desarrollo de un sistema de aprendizaje
automático está el apartado de validación y pruebas, es aquí donde se ve si
realmente el trabajo llevado a cabo ha sido fructífero. Y más importante aún, se
aprecia que cambios y desarrollos son los que mayor impacto han tenido.

Al final se crearon un total de 23 modelos, 10 de ellos basados en una red
neuronas recurrente mientras que el resto están basados en \gls{BERTa}. De los 23
modelos finales, 5 de ellos cuentan con una función de activación de salida
softmax mientras que el resto cuenta con una sigmoid. Los modelos restantes se
dividen en tres grupos dependiendo del conjunto de datos con los que fueron
entrenados, están los que se entrenaron con los datos resultados del aumento
multi-etiqueta, aquellos entrenados únicamente con los textos extraídos de
\gls{IISDa} \cite{IISDHomepage}, la última página web mencionada, y por
último aquellos que fueron entrenados juntando todos estos conjuntos, a
excepción del aumentado que se descartó por motivos de consistencia. 

En total se han llevado a cabo tres tipos de pruebas, la mas extensa y fiable de
ellas se ha realizado sobre el conjunto de prueba seleccionado. Otro de ellos se
realizó clasificando textos extraídos manualmente y viendo si los resultados
cuadran con textos correctamente y finalmente se han clasificado multitud de
abstracts de publicaciones científicas alojados en la página \gls{WebOfSciencea}, estos fueron
extraídos siguiendo una serie de consultas publicadas en la literatura, las
cuales afirmaban seleccionar trabajos de únicamente un \gls{ODSa} cada una.

\section{Conjunto de pruebas}
Las métricas expuestas en este apartado han sido calculadas clasificando los
textos del conjunto de pruebas final, la mayoría de los cuales no han sido
vistos por ningún modelo anteriormente. Esto se debe a la naturaleza secuencial
del desarrollo, en la última fase se juntaron todos los textos extraídos y se
realizó una división 75/25 entre el conjunto de entrenamiento y el de pruebas. Es
probable que alguno de los textos usados para el entrenamiento en modelos
anteriores acabaran en el conjunto de pruebas final, pudiendo alterar los datos
recogidos del mismo. La probabilidad de que esto ocurra con una magnitud tal
como para alterar estas métricas de una manera significativa es relativamente
baja por lo que estas se asumen como verdaderas y fiables.

En este apartado se muestran las siguientes métricas: precisión, exhaustividad,
exactitud y valor-F1, estas se calcularon en base a la cantidad de verdaderos
positivos (tp), verdaderos negativos (tn), falsos positivos (fp) y falsos
negativos (fn) predecidos por el modelo de la siguiente manera: 

\begin{center}
    $precision=\frac{tp}{tp + fp}$

    $exhaustibidad=\frac{tp}{tp + fn}$

    $exactutud=\frac{tp+tn}{tp + fp + tn + fn}$

    $valor-F1=2*\frac{precision * exhaustibidad}{precision + exhaustibidad}$
\end{center}

El conjunto de estas métricas puede revelar la correcta funcionalidad de un
modelo en tareas de clasificación de una manera profunda y completa ya que se
cubre todo el rango de posibilidades. 

Con la \textit{precisión} se calcula la capacidad que tiene el modelo para
predecir clases correctas, esto es, que fracción de las clases que se asignan
como positivas lo son de verdad. 

Con la \textit{exhaustividad} se comprueba lo contrario que con la precisión,
esto es cuántos de los datos que son positivos son clasificados como tal por el
modelo. De esta manera se puede saber que fracción de los datos no es
clasificada por el modelo.

Con el \textit{valor-F1} se combinan las métricas de precisión y exhaustividad
en una sola. Tiene especial importancia si ambas métricas son igual de
relevantes, afirmación que no siempre es verdadera, dependerá exclusivamente del
objetivo y funcionalidad esperada del modelo. 

Finalmente la \textit{exactitud} mide la fracción de clases que han sido
asignadas correctamente, tanto las positivas como las negativas. La fiabilidad
de esta métrica depende de varios factores, uno de los cuales es en nivel de
balanceo de las clases, si estas están desbalanceadas esta métrica se podrá
inflar clasificando la clase mayoritaria como verdadera siempre. 

En el caso de este proyecto, pese a que se han calculado todas las métricas, las
más importantes son la exactitud y el valor-F1. Se han seleccionado estas dos ya
que en conjunto representan todos los aspectos relevantes del modelo. La
exactitud es muy útil y es ampliamente usada como métrica pero hay que usarla
con cuidado y en caso de duda contrastarla con el valor-F1. Esto es debido a
que, como se ha mencionado, depende del nivel de balanceo de los datos usados.
En el caso de los conjuntos de entrenamiento y pruebas, y como se puede ver en
la TABLA \ref{table:Metricas estadisticas de la base de datos final}, se
obtiene, en conjunto, una desviación estándar de 466. Siendo esta relativamente
alta para el número de datos total usados, representados en la TABLA
\ref{table:Numero de datos por objetivo}. Como se aprecia en dicha tabla, las
clases más desbalanceadas son la del objetivo 17, contando con 2521 datos
totales, y la del objetivo 4, contando con 552, Más del doble de la media de
1070.65 en el primer caso y aproximadamente la mitad en el segundo. Si se
obvian estas dos clases, se obtiene una desviación media de 291.53, un valor
mucho mas razonable.

Es debido a ese desbalanceo por lo que se ha seleccionado el valor-F1 como una
métrica relevante. De esta manera se puede validar la fiabilidad de la exactitud
obtenida por un modelo analizando si esta es fruto de una clasificación
defectuosa y sesgada y realizar, de esta manera, un análisis más profundo de los
mismos. 

\begin{table}[H]
    \begin{tabular}{| c | c |}
        \hline
        Modelo & Ocurrencias\\
        \hline \hline
        ODS1  & 791\\ \hline
        ODS2  & 925\\ \hline
        ODS3  & 1121\\ \hline
        ODS4  & 552\\ \hline
        ODS5  & 787\\ \hline
        ODS6  & 775\\ \hline
        ODS7  & 875\\ \hline
        ODS8  & 1076\\ \hline
        ODS9  & 991\\ \hline
        ODS10 & 925\\ \hline
        ODS11 & 750\\ \hline
        ODS12 & 1127\\ \hline
        ODS13 & 1969\\ \hline
        ODS14 & 940\\ \hline
        ODS15 & 1218\\ \hline
        ODS16 & 858\\ \hline
        ODS17 & 2521\\ \hline
        \textbf{TOTAL} & \textbf{18201}\\ \hline
    \end{tabular}
    \caption{Número de datos por objetivo}
    \label{table:Numero de datos por objetivo}
\end{table}

\begin{table}[H]
    \begin{tabular}{| c | c |}
        \hline
        Métrica & Valor\\
        \hline \hline
        Media            & 1070.65\\ \hline
        Mediana          & 925.0\\ \hline
        Desviación media & 466.86\\ \hline
    \end{tabular}
    \caption{Métricas estadísticas de la base de datos final}
    \label{table:Metricas estadisticas de la base de datos final}
\end{table}

A continuación se muestra la evolución de la métricas de los diferentes modelos
a medida que se va aumentando el valor del umbral, este valor, como se
definió en la solución propuesta, representa la facilidad con la
que el modelo clasificará más de una clase a un mismo texto, siendo un 0 el caso
extremo en el que se clasificarán todas las clases y un 1 el caso en el que,
para los modelos con función de activación softmax solo se asignará una etiqueta
a un texto, y para los modelos con función de activación sigmoid no se
clasificarán casi ninguna etiqueta ya que es muy raro que esta tenga un 100\% de
posibilidad asignada. 

A continuación se muestran las métricas obtenidas por los diferentes modelos
sobre el conjunto de datos de prueba final. Destacar la importancia de
analizar el eje de abscisas en cada gráfica, puesto que este empieza en valores
diferentes dependiendo de la gráfica, por lo que dos gráficas similares pueden
variar en magnitud.

\subsection{Métricas redes recurrentes}En primer lugar se muestran los datos
obtenidos por los modelos basados en redes recurrentes.

Los resultados de estas pruebas son bastantes similares entre sí, variando
únicamente y de manera leve en magnitud por lo que a continuación se incluyen
las dos gráficas más relevantes, el resto se obvian por motivos de reiteración. 

En estas dos gráficas, \cref{Metricas modelo 1,Metricas modelo 2}, se puede apreciar
una distribución ciertamente horizontal, con poca variación de las métricas en
relación al valor del umbral. Adicionalmente se observa un valor medio del 70\%
de exactitud, de 40-50\% de precisión y de menos de 30\% de exhaustividad y
valor-F1. Esto es lo que se esperaria de estos modelos ya que están diseñados y
entrenados con datos con una sola etiqueta, los denominados como
\textit{one-hot} y no se espera un buen rendimiento de ellos en tareas de
clasificación multi etiqueta. Adicionalmente un valor tan bajo de precisión
indica una capacidad pobre de los modelos para asignar todos los objetivos
relacionados con un texto. De todas formas consiguen un valor aceptable de
exactitud por lo que sirven como una buena base sobre la que desarrollar modelos
más potentes y comparar los resultados. 


\begin{figure}[H]
    \centering
    \includegraphics[width=0.751\textwidth]{model1\_metrics}
    \captionsetup{justification=centering}
    \caption{Métricas modelo 1}
    \label{Metricas modelo 1}
\end{figure}
% \hfill
\begin{figure}[H]
    \centering
    \includegraphics[width=0.751\textwidth]{imagenes/model2\_metrics.eps}
    \captionsetup{justification=centering}
    \caption{Métricas modelo 2}
    \label{Metricas modelo 2}
\end{figure}

\subsection{Métricas modelos BERT}A continuación se muestran las métricas de
aquellos modelos basados en \gls{BERTa}.

Estos modelos a su vez se han dividido en tres grupos dependiendo del conjunto
de datos que se ha usado para entrenarlos. 

\subsubsection[short]{Datos iniciales - \textit{one-hot}}
En primer lugar se encuentran aquellos entrenados con el primer conjunto de
todos, aquel que cuenta únicamente con una etiqueta por dato.

De todos los modelos entrenados con estos datos, los primeros a analizar con
aquellos con función activación sofmax, estos son los modelos 11-13-14.

Como se puede apreciar en las gráficas, \cref{Metricas modelo 11,Metricas modelo
13,Metricas modelo 14}, se ve como los modelos tienen un rendimiento estable, es
decir, este no varía casi con el umbral. Esto es debido a la función de
activación usada ya que esta hace que no suelan asignar muchas clases a las
entradas, obteniendo así una exhaustividad relativamente baja y esta no varia al
pasar a valores de umbral mayores que un 0.3 puesto que, pasado este punto, la
diferencia entre los porcentajes precedidos por el modelo es tal que no se
suelen asignar muchos. Esto nos indica que las salidas de estos modelos suelen
ser correctas, pero también se suelen dejar objetivos asociados sin asignar.
Esto entra en lo que cabía esperar de este tipo de modelos.

\begin{figure}[H]
    \centering
    \includegraphics[width=0.751\textwidth]{model11\_metrics}
    \captionsetup{justification=centering}
    \caption{Métricas modelo 11}
    \label{Metricas modelo 11}
\end{figure}
\begin{figure}[H]
    \centering
    \includegraphics[width=0.751\textwidth]{model13\_metrics}
    \captionsetup{justification=centering}
    \caption{Métricas modelo 13}
    \label{Metricas modelo 13}
\end{figure}
\begin{figure}[H]
    \centering
    \includegraphics[width=0.751\textwidth]{model14\_metrics}
    \captionsetup{justification=centering}
    \caption{Métricas modelo 14}
    \label{Metricas modelo 14}
\end{figure}

El resto de modelos son aquellos que cuentan con una función de activación
softmax.

En las gráficas, \cref{Metricas modelo 16,Metricas modelo 17,Metricas modelo 18},
se puede apreciar que el rendimiento de estos modelos, en la mayoría de los
casos, es similar a los anteriores, obteniendo unos buenos resultados en
precisión y exactitud pero unos resultados mediocres en exhaustividad y
valor-F1. Esto era de esperar debido a que están entrenados con el mismo
conjunto de datos por lo que aprenderán de la misma manera. Destacar que el
modelo 16, \cref{Metricas modelo 16}, tiene una gráfica con una forma diferente,
esta presenta unos resultados más equilibrados, pero a costa de disminuir la
magnitud de los mismos. Esto es debido a que fue diseñado con una función de
loss\footnote{Una función de pérdida que mide la discrepancia entre las predicciones de un modelo y los valores reales durante el entrenamiento.} diferente, siendo esta poco adecuada para entrenar
modelos con función de activación softmax, en concreto se usó
\textit{Categorical cross entropy}, estando esta función diseñada para datos con
una codificación \textit{one-hot} como los usados con funciones de activación
sigmoid.
\begin{figure}[H]
    \centering
    \includegraphics[width=0.751\textwidth]{model16\_metrics}
    \captionsetup{justification=centering}
    \caption{Métricas modelo 16}
    \label{Metricas modelo 16}
\end{figure}
% \hfill
\begin{figure}[H]
    \centering
    \includegraphics[width=0.751\textwidth]{model17\_metrics}
    \captionsetup{justification=centering}
    \caption{Métricas modelo 17}
    \label{Metricas modelo 17}
\end{figure}
\begin{figure}[H]
    \centering
    \includegraphics[width=0.751\textwidth]{model18\_metrics}
    \captionsetup{justification=centering}
    \caption{Métricas modelo 18}
    \label{Metricas modelo 18}
\end{figure}



\subsubsection{Datos aumentados - multi-etiqueta}
Los siguientes modelos fueron entrenados con el conjunto de datos aumentado:

En las gráficas, \cref{Metricas modelo 19,Metricas modelo 20,Metricas modelo 21}, se
puede apreciar que, pese a una gran cantidad de datos nuevos multi-etiquetados
no obtiene unos resultados generales mucho superiores en cuanto a la magnitud de
los mismos, de todas formas estos son más equilibrados, siendo superiores en las
tareas de clasificación multi-etiqueta, como se observa en las métricas de
exhaustividad y valor-F1, estando estas entre 10 y 20 puntos porcentuales por
encima de los anteriores modelos. Adicionalmente con valores altos de umbral se
consigue una precisión extremadamente alta, estando cerca del 100\%, indicando
esto que las etiquetas a las que se les asigna una probabilidad muy alta son
en su mayoría correctas.

\begin{figure}[H]
    \centering
    \includegraphics[width=0.751\textwidth]{model19\_metrics}
    \captionsetup{justification=centering}
    \caption{Métricas modelo 19}
    \label{Metricas modelo 19}
\end{figure}
% \hfill
\begin{figure}[H]
    \centering
    \includegraphics[width=0.751\textwidth]{model20\_metrics}
    \captionsetup{justification=centering}
    \caption{Métricas modelo 20}
    \label{Metricas modelo 20}
\end{figure}
\begin{figure}[H]
    \centering
    \includegraphics[width=0.751\textwidth]{model21\_metrics}
    \captionsetup{justification=centering}
    \caption{Métricas modelo 21}
    \label{Metricas modelo 21}
\end{figure}


\subsubsection{Datos extraídos de publicaciones en \textbf{iisd} -
multi-etiqueta} \label{ch: Datos iisd} A continuación se creo una nueva base de
datos extrayendo las publicaciones de \gls{IISDa}, esta tiene una naturaleza multi-etiqueta, se entrenó un
único modelo que obtuvo unos resultados realmente buenos y equilibrados como se
observa en las métricas de la gráfica, \cref{fig:Metricas modelo 21}. Es esta se
puede apreciar como por primera vez se obtiene unos valores buenos en las
métricas de exhaustividad y valor-F1. Adicionalmente se observa que para un
valor del umbral de al rededor de 0.25 se obtienen unas métricas realmente
buenas en general. Esto puede parecer tentador pero se aprecia que el valor del
la precisión es más bajo de lo habitual, por lo que estas métricas tan buenas de
valor-F1 y de exhaustividad se obtienen a causa de una precisión más baja, por
lo que se asignan más objetivos de los adecuados. Esto se considera menos ideal
que el caso contrario, en el que se asignan menos etiquetas pero con la garantía
de que estas si están relacionadas.

Destacar que este modelo ha sido entrenado con algunos de los datos usados para
las pruebas por lo que las métricas pueden estar ciertamente infladas pero
independientemente de esto las métricas obtenidas son positivas.
\begin{figure}[H]
    \centering
    \includegraphics[width=0.751\textwidth]{model22\_metrics}
    \captionsetup{justification=centering}
    \caption{Métricas modelo 21}
    \label{fig:Metricas modelo 21}
\end{figure}

\subsubsection{Datos finales - multi-etiqueta}
Finalmente se generaron una serie de modelos con la base de datos conjunta,
estos modelos son los únicos que no se han entrenado con ninguno de los datos
incluidos en el conjunto de pruebas usado para el cálculo de las métricas, por
lo que estas son las mas fiables de todos los modelos. 

Observando las gráficas, \crefrange{Metricas modelo 23}{Metricas modelo 26}, se
aprecia que estas presentan distribuciones similares, contado con unos resultados
medios en todas las métricas para umbrales bajos y estas se van dividiendo a
medida que este aumenta, siendo el punto medio óptimo aquel asociado a un valor
del umbral de 0.5, apreciable en mayor medida en la \cref{Metricas modelo 26},
donde todas las métricas menos la de la exactitud se cruzan en este valor, para
el resto de modelos esta afirmación es, en cierta medida, arbitraria pero
indudablemente es un punto óptimo para la mayoría.

\begin{figure}[H]
    \centering
    \includegraphics[width=0.751\textwidth]{model23\_metrics}
    \captionsetup{justification=centering}
    \caption{Métricas modelo 23}
    \label{Metricas modelo 23}
\end{figure}
% \hfill
\begin{figure}[H]
    \centering
    \includegraphics[width=0.751\textwidth]{model24\_metrics}
    \captionsetup{justification=centering}
    \caption{Métricas modelo 24}
    \label{Metricas modelo 24}
\end{figure}
\begin{figure}[H]
    \centering
    \includegraphics[width=0.751\textwidth]{model25\_metrics}
    \captionsetup{justification=centering}
    \caption{Métricas modelo 25}
    \label{Metricas modelo 25}
\end{figure}
\begin{figure}[H]
    \centering
    \includegraphics[width=0.751\textwidth]{model26\_metrics}
    \captionsetup{justification=centering}
    \caption{Métricas modelo 26}
    \label{Metricas modelo 26}
\end{figure}

Como conclusión, en estas gráficas, \crefrange{Metricas modelo 11}{Metricas modelo 26}, se puede apreciar que para un valor medio del umbral se encuentra el
punto de mayor equilibro entre las métricas. Es habitual encontrar afirmaciones
indicando que un valor de 0.5 es lo adecuado cuando se trabaja con salidas
sigmoid y una clasificación multi-etiqueta. Esto se ve reforzado visualizando la
\cref{Metricas modelo 26} ya que tres de las 4 métricas se cruzan exactamente en
este valor. Para el resto no resulta tan fácil apoyar esta afirmación pero si se
priorizan los valores de precisión y en menor medida, el valor-F1, parece que
otra vez este valor de umbral es el que consigue unos resultados equilibrados. 

Cuando se habla de las funciones de activación softmax este valor se ve menos
claro, observando las gráficas podría parecer intuitivo darle un valor
relativamente bajo pero esto no parece una decisión adecuada ya que se estaría
siendo muy poco restrictivo sin ningún resultado positivo aparente reflejado en
las métricas. Debido a esto y por razones de simplicidad se usará un umbral de
0.5 para todos los modelos, apoyado en el hecho de que las gráficas,
\crefrange{Metricas modelo 11}{Metricas modelo 14}, no presentan apenas variación en
las métricas en relación al umbral.

\subsection{Métricas finales}
Finalmente, seleccionando un umbral medio de 0.5 y calculando la exactitud y
valor-F1 de todos los modelos sobre el conjunto definido de pruebas se obtienen
las siguientes gráficas de exactitud, \cref{fig:Exactitud de todos los modelos},
y valor-F1, \cref{fig:Valor-F1 de todos los modelos}, comparando el rendimiento
de todos los modelos.

Destacar que esta primera gráfica, \cref{fig:Exactitud de todos los modelos},
tiene el eje de ordenadas, representando el valor de exactitud, truncado,
iniciando este en 0.5. Esto es debido a que las diferencias entre los modelos no
son tan grandes como para ser apreciables en una escala más grande.

\begin{figure}[H]
    \centering
    \includegraphics[width=0.8\textwidth]{models\_accuracy\_metrics}
    \captionsetup{justification=centering}
    \caption{Exactitud de todos los modelos}
    \label{fig:Exactitud de todos los modelos}
\end{figure}

\begin{figure}[H]
    \centering
    \includegraphics[width=0.8\textwidth]{models\_f1\_metrics}
    \captionsetup{justification=centering}
    \caption{Valor-F1 de todos los modelos}
    \label{fig:Valor-F1 de todos los modelos}
\end{figure}

En la gráfica, \cref{fig:Exactitud de todos los modelos}, se puede observar que la
exactitud obtenida por los diferentes modelos es relativamente parecida,
rondando en torno al 80\% en la mayoría de los casos, con alguna que otra
excepción. Tendencia que se mantiene al analizar el valor-F1, en la
\cref{fig:Valor-F1 de todos los modelos} aunque en menor magnitud, rondando en
este caso un valor del 50\%. 

También se puede observar que los modelos con función de activación sigmoid pero
entrenados con el primer conjunto de datos obtienen un rendimiento peor que los
modelos entrenados con los mismos datos pero con función de activación softmax.
Esto puede ser resultado del tipo de datos usados ya que la función softmax está
diseñada para trabajar con ese tipo de etiquetado, \textit{one-hot}, mientras
que sigmoid no. Esta tendencia se aprecia en ambas gráficas \cref{fig:Exactitud
de todos los modelos,fig:Valor-F1 de todos los modelos}. Destacar las métricas
obtenidas por el modelo 16, este obtiene un valor de exactitud mediocre, del
60\%, siguiendo el patrón esperado, pero un valor-F1 superior al resto de modelos
entrenados con el mismo conjunto de datos. Esto cuadra con lo visto en la
gráfica \cref{Metricas modelo 16}, donde se aprecia que el modelo obtiene unos
datos menores pero más equilibrados.

Se puede observar una ligera mejora en ambas métricas al usar datos
multi-etiquetados y función de activación sigmoid, pero en la mayoría de los
casos esta diferencia no es tan notable. De estos modelos el que mejor exactitud
presenta es, con diferencia, el 23, obteniendo un 89\%, seguido por el 22
obteniendo un valor ligeramente menor.

Estas tendencias se ven replicada en la gráfica \cref{fig:Valor-F1 de todos los
modelos}, done se aprecia que el valor-f1 sigue el mismo patrón pero en este
caso las diferencias entre los modelos son más drásticas, es debido a esto por
lo que el eje de ordenadas empieza en un valor de 0, a diferencia de
\cref{fig:Exactitud de todos los modelos}. Esto hace de modelo 23 un modelo
aún más superior al resto, puesto que obtiene un valor-F1 significativamente
superior al resto, estando cerca del 80\%. 

Destacar el buen rendimiento del modelo 22. Esto difiere con lo que se
esperaria, ya que ha sido entrenado con menos datos que los posteriores a el.
Esto, como ya se he explicado en \cref{ch: Datos iisd} es debido a que el modelo
ha sido entrenado con una parte de los datos usados para realizar las pruebas,
es por esto que las métricas están infladas y no representan del todo el
rendimiento del modelo.

\subsection{Conclusiones parciales}
Como conclusión final extraída de los resultados de estas pruebas comentar que
el hecho de usar técnicas de aumentado de datos para transformar un conjunto de
datos de tipo \textit{one-hot} en uno multi-etiquetado resultó en un impacto
positivo en el rendimiento de los modelos, mientras que entrenar un modelo con
función de activación sigmoid con datos no multi-etiquetados no da buenos
resultados. Destacar también la importancia de separar los conjuntos de
entrenamiento y de pruebas, ya que las métricas infladas resultantes pueden
generar conclusiones erróneas y es una practica poco recomendable y que hay que
llevar a cabo con cautela. Finalmente destacar la importancia de entrenar varios
modelos, al menos con conjuntos de datos y problemas pequeños, donde los tiempos
de entrenamiento y los recursos necesarios no son muy altos ya que la
aleatoriedad del entrenamiento y los pesos iniciales tiene un impacto alto, como
se aprecia en las gráficas \cref{fig:Exactitud de todos los modelos,fig:Valor-F1
de todos los modelos}, donde el modelo 23 obtiene unos resultados
significativamente superiores al resto, aún siendo entrenado en las mismas
condiciones que el resto y con los mismos datos que aquellos posteriores a el. 

\section{Textos seleccionados}
Una parte esencial de la validación del rendimiento de un modelo es analizar las
salidas generadas sobre datos seleccionados manualmente para analizar, de una
manera más directa como este procesa las entradas y ``piensa''. De esta manera
se puede ver que palabras, frases y patrones hacen que el modelo tome ciertas
decisiones, análisis que, como se explicará más adelante, puede resultar crucial
en el ultimo apartado de la validación de los modelos.

Se han elegido manualmente una serie de textos para clasificar con el modelo que
mejores resultados obtuvo en las pruebas anteriores, el 23. De esta manera se
puede apreciar de una manera directa, como este funciona y razona y se puede
analizar si es capaz de interpretar correctamente los textos.

Estos textos son fragmentos del abstract de una serie de publicaciones
científicas escogidas manualmente de Scopus tras una primera búsqueda
relacionada con algún tema u objetivo en concreto. No se presentan textos
relacionados con todos los objetivos ya que se consideró que no aportaria
información relevante analizar aquí en profundidad la literatura referente a
todos.

A continuación se presentan los textos seguidos de una tabla conteniendo los
tres objetivos con un mayor porcentaje asignado. No se han incluido los
asociados a más objetivos porque llegada una determinada magnitud, el porcentaje
de pertenencia deja de tener sentido.

Se han realizado estas búsquedas como base sobre la que extraer textos
relevantes, relacionados con sostenibilidad. No se espera que sean clasificados
acorde a ningún objetivo en concreto, únicamente en base a la búsqueda, si no
que esto dependerá íntegramente del texto extraído. 

\subsection{Textos Sostenibilidad y sanidad}
Estos textos han sido sacados de los abstracts resultantes al realizar la
siguiente búsqueda en \textit{Scopus}: sostenibilidad y sanidad

De entre todas las publicaciones resultantes de la búsqueda se eligieron 3 al
azar. Teniendo en cuenta la búsqueda realizada se esperaria que el modelo los
clasificase como relacionados con el objetivo 3.

\subsubsection{Texto 1}
\begin{center}
    \textit{``Background: Suriname is a uppermiddle-income country with a relatively
    high prevalence of preventable pregnancy complications. Access to and usage
    of high-quality maternity care services are lacking. The implementation of
    group care (GC) may yield maternal and child health
    improvements.''}\cite{validationtexts1}
    \begin{table}[H]
        % \begin{center}
                    % \resizebox{\textwidth}{!}{
            \begin{tabular}{c | c | c }
                \hline
                ODS3 & ODS17 & ODS2\\ \hline
                0.524 & 0.041 & 0.032\\ \hline
            \end{tabular}%}
            \caption{Porcentajes texto 1 - sanidad}
        % \end{center}
    \end{table}
\end{center}

El texto extraído habla sobre la importancia del acceso a servicios sanitarios
por parte de las mujeres embarazadas para mejorar la salud de las madres y los
infantes. 

Como se esperaba, el modelo le asigna un 52\% de pertenencia al objetivo 3 ya
que se habla de sanidad. El siguiente objetivo con mas porcentaje asignado es el
17 con un 4\% y el 2 con un 3\%. Estos últimos pueden ignorarse ya que tienen un
porcentaje asignado muy bajo, al igual que el resto de objetivos.

\subsubsection{Texto 2}
\begin{center}
    \textit{``Globally, forests serve as the largest storehouses for (non)
    indigenous trees, and are essential for the ecosystems' sustenance, yet,
    increased deforestation practices associated with activities, such as tree
    logging, agriculture, and urban expansion continue to put pressure on
    existing forest areas, leading to massive land use cover
    change''}\cite{validationtexts2}
    \begin{table}[H]
        % \begin{center}
                    % \resizebox{\textwidth}{!}{
            \begin{tabular}{c | c | c }
                \hline
                ODS15 & ODS11 & ODS13\\ \hline
                0.997 & 0.047 & 0.041\\ \hline
            \end{tabular}%}
            \caption{Porcentajes texto 2 - sanidad}
        % \end{center}
    \end{table}
\end{center}

Este texto habla sobre el peligro de la deforestación sobre los ecosistemas
terrestres y como esta practica pone en peligro las especies no indígenas.

Pese a ser resultado de una búsqueda sobre sanidad, este texto está altamente
relacionado con el objetivo 15, el referente a los ecosistemas terrestres. Esto se
ve reflejado en la predicción del modelo, asignándole a este objetivo un
porcentaje del 99.7\%, mientras que al resto menos de un 5\%.

\subsubsection{Texto 3}
\begin{center}
    \textit{``With high staff vacancies in the health services, it is important
    to consider pragmatic methods of data collection for implementation
    evaluation. This paper presents a cross-sectional rapid evaluation of a
    handheld medical device designed for remote examinations, piloted in
    Northern England.''}\cite{validationtexts3}
    \begin{table}[H]
            \begin{tabular}{c | c | c }
                \hline
                ODS3 & ODS17 & ODS16\\ \hline
                0.906 & 0.093 & 0.019\\ \hline
            \end{tabular}
            \caption{Porcentajes texto 3 - sanidad}
    \end{table}
\end{center}

Este último texto sobre sanidad si está más relacionado con el tema principal,
habla sobre un dispositivo médico diseñado para realizar examinaciones en
remoto.

Como cabe esperas el objetivo 3 tiene asignado un 90\% de pertenencia.
Adicionalmente le asigna cerca de un 10\% al objetivo 17, esto puede ser debido
a que se habla de métodos para la recogida de datos, lo cual está más
relacionado con este objetivo.

\subsection{Textos Sostenibilidad e industria}

En está segunda búsqueda se centro el enfoque en la sostenibilidad y la
industria.

No se espera ningún objetivo en concreto ya que hay muchos que pueden
relacionarse con temas industriales, pero los principales son el 7, relacionado
con energía, 8 relacionado con crecimiento económico, 9 relacionado de manera
más directa con la industria, el 11 relacionado con las ciudades y finalmente el
12 relacionado con consumo y producción responsable. 

\subsubsection{Texto 1}
\begin{center}
    \textit{``With the world's population continuing to grow exponentially, with
    many 'food deserts' across the globe, including even in rich countries, food
    production is more important than ever. Finding alternative ways to produce
    food, in a sustainable way, is increasingly important and something that is
    on the minds of scientists, engineers, policy makers, and other
    professionals.''}\cite{validationtexts4}
    \begin{table}[H]
        \begin{tabular}{c | c | c }
            \hline
            ODS2 & ODS12 & ODS15\\ \hline
            0.733 & 0.062 & 0.036\\ \hline
        \end{tabular}
        \caption{Porcentajes texto 1 - industria}
\end{table}
\end{center}

Este primer texto se escapa ciertamente del tema principal del apartado, la
industria. Hablando de los llamados desiertos de comida y la importancia de la
producción e industrie alimenticia, y formas de hacerlas más sostenibles.

Leyendo el texto se ve la relación directa con el objetivo 2, debido a que la
comida es su tema principal. También se encuentra el objetivo 12, aunque con un
porcentaje bajo, del 6\%. Esto es menos de lo que se esperaría del texto ya que
se habla de como conseguir una producción sostenible de comida. Esto puede ser
resultado de la corta longitud del texto. 


\subsubsection{Texto 2}
\begin{center}
    \textit{``With rising pollution emissions, it is vital to devise regulatory
    policies that ensure sustainable development. Green innovation offers an
    alternative strategy, fostering economic progress and environmental
    sustainability. While existing literature supports the positive role of
    green innovation in firm-level decisions, its specific impact on alleviating
    environmental regulation pressures remains
    unexplored.''}\cite{validationtexts5}
    \begin{table}[H]
        \begin{tabular}{c | c | c }
            \hline
            ODS13 & ODS12 & ODS9 \\ \hline
            0.164 & 0.0888 & 0.056\\ \hline
        \end{tabular}
        \caption{Porcentajes texto 2 - industria}
\end{table}
\end{center}

Este último texto del apartado de industria habla de la importancia de crear
políticas regulatorias en referencia al aumento de emisiones. El principal texto
predecido es el 13, como se esperaría hablando de este tema, pero este tiene
asignado un porcentaje bajo. Esto puede ser, de nuevo, debido a la longitud del
texto. 

\subsection{Textos Sostenibilidad e igualdad}

En este apartado se han buscado textos relacionados con sotenibilidad e
igualdad. Temas mas afines al objetivo 5 y 10.

\subsubsection{Texto 1}
\begin{center}
    \textit{``It is hard to establish whether a company supports internal
    sustainability efforts (ISEs) like gender equality, diversity, and general
    staff welfare, not least because of a lack of methodologies operationalizing
    these internal sustainability practices, and of data honestly documenting
    such efforts. We developed and validated a six-dimension framework
    reflecting Internal Sustainability Efforts (ISEs), gathered more than 350K
    employee reviews of 104 major companies across the whole US for the
    (2008-2020) years, and developed a deep-learning framework scoring these
    reviews in terms of the six ISEs.''}\cite{validationtexts6}
    \begin{table}[H]
        \begin{tabular}{c | c | c }
            \hline
            ODS5 & ODS12 & ODS17 \\ \hline
            0.459 & 0.253 & 0.201\\ \hline
        \end{tabular}
        \caption{Porcentajes texto 1 - igualdad}
\end{table}
\end{center}

Este primer texto habla sobre las dificultades a la hora de esclarecer si una
empresa está adoptando medidas a favor de la sostenibilidad, entre ellas
igualdad de género y como su sistema desarrollado ayuda a determinarlo. 

El principal objetivo clasificado por el modelo es el 5, con un 46\% asignado.
Esto es lo que se podía esperar leyendo el texto ya que este tiene la igualdad
de género como tema principal. También le asigna un 20\% a los objetivos 12 y 7,
metricas que no se esperarían de este texto, más aún cuando la única mención
posible relacionada es la que se hace a los esfuezos por la sostenibilidad, pero
esta se hace en un caracter cerrado como se menciona a continuación de la misma
en el texto.

\subsubsection{Texto 2}
\begin{center}
    \textit{``Our aim was to evaluate how well we carried out authentic
    co-creation of an intervention to support midwives have a dialogue about
    alcohol consumption with pregnant women. Patient involvement: Recent
    maternity service users including women with experience of harm due to
    alcohol during pregnancy provided feedback on the design, conduct and
    dissemination of the study.''}\cite{validationtexts7}
    \begin{table}[H]
        \begin{tabular}{c | c | c }
            \hline
            ODS5 & ODS3 & ODS2\\ \hline
            0.424 & 0.287 & 0.076 \\ \hline
        \end{tabular}
        \caption{Porcentajes texto 2 - igualdad}
\end{table}
\end{center}

Este texto habla de una iniciativa para concienciar a mujeres embarazadas del
peligro que acarrea el consumo de alcohol durante el embarazo. 

Los objetivos con más porcentaje asignado por el modelo son el 5 y el 3,
resultados que cuadran con el texto dado. El 5 es de los dos el que menos podría
cuadrar dado que no se menciona la igualdad de género, pero siempre que se
mencioan mujeres este objetivo está presente. El único problema con los
porcentajes asignados es la magnitud de los mismos, estos son más bajos de lo
que se esperaría del modelo, lo cual sería un 50\% o más para ambos, mientras
que los reales son un 42\% y un 28\%.

\subsection{Textos Sostenibilidad y energia}
Este último grupo de textos están relacionados con al sostenibilidad y la
energía, dado esto se esperaria una mayor presencia del objetivo 7.

\subsubsection{Texto 1}
\begin{center}
    \textit{``Additive Manufacturing (AM) or 3D printing techniques use fused
    layers of the material to build cross sectional geometry of product. As
    variable processing parameters have an impact on the product quality, it is
    crucial to ascertain relationships of AM process parameters, productivity,
    sustainability, and structure performance.''}\cite{validationtexts8}
    \begin{table}[H]
        \begin{tabular}{c | c | c }
            \hline
            'ODS12' & 'ODS9' & 'ODS8' \\ \hline
            0.713 & 0.473 & 0.102\\ \hline
        \end{tabular}
        \caption{Porcentajes texto 1 - energía}
\end{table}
\end{center}

Este primer texto habla de un tipo de manufacturación aditiva y los impactos en
sostenibilidad, productividad y rendimiento que tiene. 

Las predicciones de este modelo son lo que se podría esperar de un texto así,
una mayor presencia del objetivo 12, con un 71\% de pertenencia ya que se habla
de sotenibilidad en un entorno industrial y de consumo y el 9, con un 47\%, que,
pese a quedarse por debajo del umbral, al ser la diferencia de 3 puntos
porcentuales se puede considerar como asignado para motivos de análisis. La
asignación de este último objetivo resulta esperable y adecuada estando este
relacionado con la industria y la innovación en la misma. 


\subsubsection{Texto 2}
\begin{center}
    \textit{``Electric Vehicles are a suitable solution for sustainability in
    transportation applications. The era of wireless power transfer in free
    space began with the evolution of Tesla coils being energized by microwaves.
    High-frequency inverters act as the electrifier for high power wireless
    charging. DC-DC converters play an indispensable role in converting the AC
    power from a high-frequency inverter to DC power to the battery or the
    energy storage system in the vehicle.''}\cite{validationtexts9}
    \begin{table}[H]
        \begin{tabular}{c | c | c }
            \hline
            ODS9 & ODS7 & ODS11\\ \hline
            0.261 & 0.147 & 0.090 \\ \hline
        \end{tabular}
        \caption{Porcentajes texto 2 - energía}
\end{table}
\end{center}

Este texto habla sobre la transferencia inalámbrica de potencia eléctrica y del
papel de los coches eléctricos como solución sostenible a los problemas de
transporte.

Los porcentajes asignados a este texto son muy bajas, probablemente debido a la
cantidad de tecnicismos usados y a que estos no están directamente relacionados
con el vocabulario que se esperaría en un contexto de sostenibilidad. De todas
formas los objetivos con mayor porcentaje si que son los que se espera de un
texto como este, siendo el 9, con un 26\% y el 7 con un 15\%, estando el primero
relacionado con la industria y el segundo con la energía eléctrica.

\subsubsection{Texto 3}
\begin{center}
    \textit{``Aquaculture is a form of agriculture that is practised in the
    Sistan region. This practice not only enhances water usage efficiency but
    also generates additional income for the local farmers. The extensive
    dependence of the aquaculture sector on energy and chemicals poses a
    jeopardises environmental stability and endangers the sustainability of
    production in the long run.''}\cite{validationtexts10}
    \begin{table}[H]
        \begin{tabular}{c | c | c }
            \hline
            ODS6 & ODS12 & ODS7\\ \hline
            0.315 & 0.251 & 0.230\\ \hline
        \end{tabular}
        \caption{Porcentajes texto 3 - energía}
\end{table}
\end{center}

Este último texto habla sobre la práctica denominada \textit{Aquaculture} y como
esta afecta al consumo de agua y a la sotenibilidad.

Este texto es de los más ambiguos, de todas formas si se analizan los objetivos
predecidos estos tienen bastante sentido, siendo el que mayor porcentaje ha
obtenido el 6, con un 32\%, seguido del 12 con un 25\% y el 7 con un 23\%. Estos
están correctamente asignados, estando el primero relacionado con el consumo de
agua, el segundo con consumo y producción sostenible y el último con la energía.

\subsection{Conclusiones parciales}
Como conclusión de estos análisis destacar la importancia de la aparición de
ciertas palabras en el texto, como por ejemplo \textit{mujer} a la hora de
clasificar el objetivo 5. 

Adicionalmente destacar la excelente capacidad del modelo a la hora de
identificar aparte del tema principal de un texto, temas secundarios que son
mencionados de manera superficial. 

\section{Literatura científica}

La ultima validación de los modelos se realizó usando un corpus de literatura
científica, en concreto los abstracts de publicaciones realizadas en
\gls{WebOfSciencea}. Dichos abstracts fueron extraídos por medio de unas
consultas publicadas en un trabajo de la universidad de Estocolmo \cite{SDGQueries}. En
dicho trabajo los autores proporcionan 17 de estas consultas especializadas,
cada una en un objetivo, para resultar en publicaciones relacionadas con cada
uno en concreto. La fiabilidad de estas métricas se discutirá más adelante pero
sirven como una buena base para analizar el rendimiento de los modelos en un
entrono similar al final.

Se extrajeron un total de 1000 publicaciones de cada objetivo, siendo este el
máximo permitido por la propia plataforma, destacar que no todas las
publicaciones vienen con abstract incluido por lo que el número de textos
clasificados es menor de 1000. Posteriormente los abstracts de dichas
publicaciones se clasificaron usando el modelo y se contó el numero total de
objetivos asignados a cada consulta. Se espera que haya cerca de 1000 objetivos
asignados correctamente por consulta.

\subsection{Modelos de redes recurrentes}
De todos los modelos generados se muestran los resultados de los más relevantes,
\cref{table:Clasificaciones de cada modelo recurrente por objetivo}, esto se
calculó realizando la media de etiquetas asignadas entre los 17 objetivos por
cada modelo, representadas en \cref{table:Media de clasificaciones por modelo
recurrente}, y seleccionando los 4 mejores, estos coinciden con los 4 primeros.

\begin{table}[H]
    \begin{tabular}{| c | c |}
        \hline
        Modelo & Media de clasificaciones\\
        \hline \hline
        1  & \textbf{165.82}\\ \hline
        2  & \textbf{139.94}\\ \hline
        3  & \textbf{149.76}\\ \hline
        4  & \textbf{134.29}\\ \hline
        5  & 100.88\\ \hline
        6  & 113.29\\ \hline
        7  & 108.70\\ \hline
        8  & 85.94\\ \hline
        9  & 97.58\\ \hline
        10 & 74.58\\ \hline        
    \end{tabular}
    \caption{Media de clasificaciones por modelo recurrente}
    \label{table:Media de clasificaciones por modelo recurrente}
\end{table}

En la tabla, \cref{table:Clasificaciones de cada modelo recurrente por objetivo}
se muestra el número de objetivos asignados por cada modelo basado en una red
recurrente a todos los textos científicos extraídos, en ella se puede apreciar
que el número es relativamente bajo para todos, siendo el que mayor rendimiento
muestra el modelo 1, aunque esto no se ve reflejado en todos los objetivos.
Demostrando así la poca fiabilidad de estos modelos a la hora de realizar un
estudio cuantitativo ya que, aunque no se espere que se clasifiquen los 1000
textos de manera correcta, se obtienen unos resultados muy pobres, clasificando,
en el mejor de los casos, 576 textos y de media 117 por objetivo, estando muy
lejos de los 1000.

\begin{table}[H]
    \begin{tabular}{| c || c | c  | c  | c | c |}
        \hline
        Objetivo & Modelo 1 & Modelo2 & Modelo3 & Modelo4 & ODSsTotales\\
        \hline \hline
        ODS1  & 576 & 416 & 294 & 381 & 855\\ \hline
        ODS2  & 59  & 140 & 98  & 143 & 913\\ \hline
        ODS3  & 128 & 110 & 90  & 22  & 284\\ \hline
        ODS4  & 56  & 41  & 21  & 35  & 145\\ \hline
        ODS5  & 345 & 256 & 315 & 154 & 619\\ \hline
        ODS6  & 169 & 78  & 63  & 211 & 851\\ \hline
        ODS7  & 24  & 129 & 43  & 256 & 788\\ \hline
        ODS8  & 31  & 42  & 5   & 93  & 975\\ \hline
        ODS9  & 31  & 178 & 31  & 40  & 965\\ \hline
        ODS10 & 81  & 54  & 64  & 67  & 831\\ \hline
        ODS11 & 328 & 233 & 175 & 341 & 987\\ \hline
        ODS12 & 231 & 16  & 67  & 173 & 830\\ \hline
        ODS13 & 99  & 134 & 566 & 21  & 914\\ \hline
        ODS14 & 18  & 115 & 171 & 58  & 983\\ \hline
        ODS15 & 169 & 224 & 320 & 62  & 978\\ \hline
        ODS16 & 247 & 44  & 196 & 173 & 777\\ \hline
        ODS17 & 227 & 169 & 27  & 53  & 994\\ \hline
    \end{tabular}
    \caption{Clasificaciones de cada modelo recurrente por objetivo}
    \label{table:Clasificaciones de cada modelo recurrente por objetivo}
\end{table}

\subsection{Modelos de BERT}

A continuación, \crefrange{table:Clasificaciones modelos
softmax}{table:Clasificaciones modelos sigmoid - datos extraidos de internet/finales} se muestran las clasificaciones realizadas por los modelos
basados en \gls{BERTa}, de estos se espera un mejor rendimiento ya que son más capaces
a la hora de entender los textos.

\subsubsection{Modelos softmax}
En primera instancia, \cref{table:Clasificaciones modelos softmax}, se muestran
las clasificaciones de los modelos con función de activación softmax, entrenados
con los mismos datos que los modelos basados en redes recurrentes,
\cref{table:Clasificaciones de cada modelo recurrente por objetivo}, se puede
observar que, en la mayoría de los casos se obtienen unos resultados muy
superiores, con unas notables excepciones, presentes en todos los modelos. Estas
son en los objetivos del 8 al 17. En estos últimos se clasifican correctamente
muy pocos textos, siendo en todos los casos menos de la mitad de los textos
totales. Esto puede ser debido a un problema en las consultas, tema abordado al
final del apartado, o un problema con los modelos. Si resulta ser un problema
con las consultas, este patrón se apreciará en el resto de tablas, si resulta
ser un problema con los modelos este patrón solo estará presente en esta tabla. 

\begin{table}[H]
    \begin{tabular}{| c || c | c  | c  | c  | c |}
        \hline
        Objetivo & Modelo 11 & Modelo13 & Modelo14 & Modelo15 & ODSsTotales\\
        \hline \hline
        ODS1  & 742 & 670 & 737 & 693 & 855\\ \hline
        ODS2  & 796 & 808 & 795 & 786 & 913\\ \hline
        ODS3  & 261 & 269 & 258 & 277 & 284\\ \hline
        ODS4  & 127 & 126 & 118 & 127 & 145\\ \hline
        ODS5  & 573 & 566 & 580 & 528 & 619\\ \hline
        ODS6  & 461 & 457 & 504 & 479 & 851\\ \hline
        ODS7  & 543 & 601 & 741 & 649 & 788\\ \hline
        ODS8  & 52  & 99  & 39  & 39  & 975\\ \hline
        ODS9  & 158 & 229 & 105 & 330 & 965\\ \hline
        ODS10 & 74  & 107 & 55  & 252 & 831\\ \hline
        ODS11 & 93  & 166 & 179 & 248 & 987\\ \hline
        ODS12 & 126 & 101 & 145 & 144 & 830\\ \hline
        ODS13 & 56  & 65  & 81  & 65  & 914\\ \hline
        ODS14 & 189 & 131 & 91  & 323 & 983\\ \hline
        ODS15 & 414 & 376 & 383 & 459 & 978\\ \hline
        ODS16 & 388 & 350 & 422 & 561 & 777\\ \hline
        ODS17 & 292 & 161 & 162 & 285 & 994\\ \hline
    \end{tabular}
    \caption{Clasificaciones modelos softmax}
    \label{table:Clasificaciones modelos softmax}
\end{table}

\subsubsection{Modelos sigmoid}
Las siguientes clasificaciones son las de los modelos con función de activación
sigmoid pero entrenados con los datos iniciales, los \textit{one-hot}, como se
puede observar en la tabla \cref{table:Clasificaciones modelos sigmoid - datos
iniciales} estos no obtienen unos resultados buenos a excepción del modelo 16
que obtiene un número de clasificaciones excelente. De todas formas, y como se
analizó en la gráfica \cref{Metricas modelo 16}, este modelo, aún obteniendo una
buena métrica de exhaustividad, tiende a asignar muchos objetivos pero con menos
acierto.

Adicionalmente la tendencia observada en los datos anteriores,
\cref{table:Clasificaciones modelos softmax}, sigue presente, aunque se aprecia
menos ya que dos de los modelos clasifican muy pocos objetivos y el otro
restante clasifica de más. De todas formas se sigue observando una reducción en
el número de objetivos clasificados a partir del 8.

\subsubsection{Entrenado con datos one-hot}
\begin{table}[H]
    \begin{tabular}{| c || c | c  | c  | c  | c |}
        \hline
        Objetivo & Modelo 16 & Modelo17 & Modelo18 & ODSsTotales\\
        \hline \hline
        ODS1   & 845 & 321 & 287 & 855\\ \hline
        ODS2   & 902 & 55  & 213 & 913\\ \hline
        ODS3   & 284 & 71  & 104 & 284\\ \hline
        ODS4   & 144 & 34  & 88  & 145\\ \hline
        ODS5   & 615 & 162 & 317 & 619\\ \hline
        ODS6   & 846 & 236 & 235 & 851\\ \hline
        ODS7   & 755 & 176 & 487 & 788\\ \hline
        ODS8   & 417 & 1   & 7   & 975\\ \hline
        ODS9   & 903 & 0   & 78  & 965\\ \hline
        ODS10  & 562 & 0   & 11  & 831\\ \hline
        ODS11  & 422 & 5   & 6   & 987\\ \hline
        ODS12  & 740 & 0   & 17  & 830\\ \hline
        ODS13  & 772 & 1   & 3   & 914\\ \hline
        ODS14  & 954 & 2   & 47  & 983\\ \hline
        ODS15  & 625 & 69  & 345 & 978\\ \hline
        ODS16  & 582 & 143 & 170 & 777\\ \hline
        ODS17  & 859 & 1   & 47  & 994\\ \hline
    \end{tabular}
    \caption{Clasificaciones modelos sigmoid - datos iniciales}
    \label{table:Clasificaciones modelos sigmoid - datos iniciales}
\end{table}


\subsubsection{Entrenado con datos aumentados}
A partir de este punto se muestran los datos obtenidos por los modelos
entrenados con datos multi-etiqueta, siendo estos primeros aquellos entrenados
con los datos iniciales aumentados.

En la tabla \cref{table:Clasificaciones modelos sigmoid - datos aumentados} se
puede apreciar un buen rendimiento general de todos los modelos, destacando el
obtenido por el modelo 21, ya que este obtiene un buen numero de clasificaciones
para todos los objetivos. Esto contrasta con el resto de métricas analizadas,
\cref{fig:Exactitud de todos los modelos,fig:Valor-F1 de todos los modelos}, en
las que este modelo no obtiene unos resultados destacables. La explicación de esta
mejora no está clara y pueden ser causada por varios factores. Destacar también
que las clasificaciones de los tres modelos para todos los objetivos anteriores
al 8 son bastante similares, destacando ligeramente el rendimiento del modelo
21, pero en una medida que cabria esperar, sobre todo siendo su valor-F1
ligeramente superior que el del resto.  

Adicionalmente indicar que le patrón analizado de una disminución en el numero
de clasificaciones a partir del objetivo 8 sigue estando presente en estas
métricas, obteniendo estos objetivos un número de clasificaciones cercano a la
mitad de los esperado en la mayoría de los casos. 

\begin{table}[H]
    \begin{tabular}{| c || c | c  | c  | c  | c |}
        \hline
        Objetivo & Modelo 19 & Modelo20 & Modelo21 & ODSsTotales\\
        \hline \hline
        ODS1   & 536 & 465 & 705 & 855\\ \hline
        ODS2   & 797 & 849 & 859 & 913\\ \hline
        ODS3   & 247 & 257 & 280 & 284\\ \hline
        ODS4   & 138 & 135 & 141 & 145\\ \hline
        ODS5   & 579 & 605 & 608 & 619\\ \hline
        ODS6   & 529 & 593 & 509 & 851\\ \hline
        ODS7   & 711 & 638 & 631 & 788\\ \hline
        ODS8   & 26  & 139 & 96  & 975\\ \hline
        ODS9   & 291 & 198 & 497 & 965\\ \hline
        ODS10  & 137 & 60  & 377 & 831\\ \hline
        ODS11  & 172 & 199 & 628 & 987\\ \hline
        ODS12  & 175 & 409 & 348 & 830\\ \hline
        ODS13  & 317 & 486 & 659 & 914\\ \hline
        ODS14  & 229 & 204 & 531 & 983\\ \hline
        ODS15  & 448 & 444 & 420 & 978\\ \hline
        ODS16  & 506 & 540 & 536 & 777\\ \hline
        ODS17  & 322 & 435 & 357 & 994\\ \hline
    \end{tabular}
    \caption{Clasificaciones modelos sigmoid - datos aumentados}
    \label{table:Clasificaciones modelos sigmoid - datos aumentados}
\end{table}


\subsubsection{Entrenado con datos aumentados extraídos de internet}
Finalmente se presentan las clasificaciones generadas por los últimos modelos,
aquellos entrenados con los datos finales, a excepción del modelo 22 que fue
entrenado únicamente con los datos multi-etiqueta extraídos de internet. 

Estos modelos, como se presenta en la tabla, \cref{table:Clasificaciones modelos
sigmoid - datos extraidos de internet/finales}, y a diferencia de lo que cabria
esperar y de lo que indica las métricas de las gráficas, \cref{fig:Exactitud de
todos los modelos,fig:Valor-F1 de todos los modelos}, presentan un rendimiento
menor que los modelos anteriores, \cref{table:Clasificaciones modelos sigmoid -
datos aumentados}. 

De todos los modelos, es el 23 el que obtiene los mejores resultados,
presentando, en este caso, una diferencia notable con el modelo 22 confirmando
así que las métricas de este último, representadas en las gráficas,
\cref{fig:Exactitud de todos los modelos,fig:Valor-F1 de todos los modelos},
estaban infladas debido a que se entrenó con parte de los datos presentes en el
conjunto de pruebas usado para calcularlas.

Destacar finalmente la reiteración del patrón identificado en el resto de tablas
anteriores, obteniendo un número de clasificaciones menor en todos los objetivos
a partir del 8. En este caso es ciertamente más sutil, con casos como el
objetivo 16, en el que se clasifican casi tantos textos como en el 7. Y el
objetivo 15, obteniendo una fracción menor de textos clasificados pero aún así
es ciertamente superior al resto de objetivos del patrón.

\begin{table}[H]
    \begin{tabular}{| c || c | c  | c  | c  | c  | c  | c |}
        \hline
        Objetivo & Modelo 22 & Modelo23 & Modelo24 & Modelo25 & Modelo26 & ODSsTotales\\
        \hline \hline
        ODS1   & 180 & 765 & 208 & 139 & 232 & 855\\ \hline
        ODS2   & 408 & 811 & 342 & 545 & 781 & 913\\ \hline
        ODS3   & 194 & 275 & 119 & 234 & 233 & 284\\ \hline
        ODS4   & 11  & 105 & 8   & 52  & 36  & 145\\ \hline
        ODS5   & 430 & 603 & 347 & 512 & 364 & 619\\ \hline
        ODS6   & 379 & 436 & 238 & 381 & 499 & 851\\ \hline
        ODS7   & 304 & 442 & 174 & 249 & 277 & 788\\ \hline
        ODS8   & 21  & 112 & 57  & 82  & 129 & 975\\ \hline
        ODS9   & 31  & 184 & 33  & 59  & 160 & 965\\ \hline
        ODS10  & 4   & 31  & 8   & 14  & 48  & 831\\ \hline
        ODS11  & 1   & 202 & 1   & 7   & 8   & 987\\ \hline
        ODS12  & 165 & 125 & 229 & 130 & 314 & 830\\ \hline
        ODS13  & 241 & 360 & 195 & 236 & 281 & 914\\ \hline
        ODS14  & 59  & 118 & 66  & 56  & 72  & 983\\ \hline
        ODS15  & 366 & 411 & 342 & 390 & 380 & 978\\ \hline
        ODS16  & 174 & 407 & 100 & 414 & 266 & 777\\ \hline
        ODS17  & 313 & 298 & 326 & 392 & 452 & 994\\ \hline
    \end{tabular}
    \caption{Clasificaciones modelos sigmoid - datos extraídos de internet/finales}
    \label{table:Clasificaciones modelos sigmoid - datos extraidos de internet/finales}
\end{table}

\subsection{Análisis de los datos}
Como se ha explicado en el apartado anterior, se ha identificado un patrón en
las clasificaciones, a partir del objetivo 8 se identifican significativamente
menos textos asignados a cada objetivo. Esto podría ser problema de los modelos
o de los datos usados para el entrenamiento, de todas formas esta posibilidad
pasa a un segundo plano ya que la presencia del mismo patrón, en mayor o menor
medida, independientemente de la arquitectura del modelo y de los datos usados
para su entrenamiento hace de esto una posibilidad remota. Es por esto por lo
que se ha decidido analizar los datos extraídos haciendo uso de la herramienta
\textit{VosViewer} \cite{VOSViewer}. Esta herramienta, a partir de un corpus de
artículos científicos exportados de bases de datos científicas como  \gls{WebOfSciencea}, genera
una gráfica de agrupación con todos los términos con mayor presencia entre los
documentos, junto con las relaciones y magnitudes de los mismos. De esta manera
se puede analizar la taxonomía de un corpus científico de manera rápida y
sencilla. 

A continuación se muestran los mapas de dispersión generados para todos los
objetivos a partir del 8, con el fin de analizar la taxonomía presente en cada
uno y esclarecer el porque del patrón. Si en estos mapas se aprecia una
taxonomía diferente a la que se esperaria para cada objetivo, serviría como
justificación para la aparición de este tipo de patrón. Adicionalmente se puede
analizar que otros objetivos se clasifican cuando se le presentan estos textos a
los modelos como validación final de los mismos ya que si clasifican los
objetivos relacionados con la taxonomía extraída se demostrará el correcto
funcionamiento de los mismos.

\subsubsection{Análisis objetivo 8}
Esta primera gráfica, \cref{Mapa de relaciones del objetivo 8}, muestra las ocurrencias y
relaciones de  palabras, términos y expresiones presentes en los textos
relacionados con el objetivo 8. Estando este objetivo relacionado con el
crecimiento económico y el trabajo decente se esperaría encontrar en mayor o
menor magnitud, términos económicos y relacionados con el trabajo.

Analizando el mapa generado por VosViewer se puede ver que los términos más
cunes son \textit{effect}, \textit{access}, \textit{management} y
\textit{health}. Siendo estos los términos principales se puede entender que los
modelos no fueran capaces de clasificar correctamente los textos  ya que son
términos muy generales y poco relacionados con el objetivo 8. Analizando más en
profundidad la figura, se puede apreciar algún término relacionado con la
economía y el trabajo como puede ser \textit{income} o \textit{employment}.
Adicionalmente analizado las agrupaciones generadas por el programa se
identifican cinco grupos diferentes, ninguno de los cuales directamente
relacionado con el objetivo 8:
\begin{itemize}
    \item \textbf{Grupo amarillo}: Este primer grupo está poco relacionado con
    los temas principales del objetivo 8, estando más directamente relacionado
    con la energía. De todas formas, estos son términos que podrían estar, en
    ocasiones, ligeramente relacionados con el crecimiento económico, siendo la
    energía parte esencial del mismo.
    \item \textbf{Grupo azul}: Este grupo es de todos el menos relacionado,
    incluyendo términos relacionados con la salud. Al igual que en el caso
    anterior son términos que podrían encontrarse a la par de temas económicos
    pero en ningún caso se esperaria encontrarlos en un primer plano.
    \item \textbf{Grupo rojo}: Este grupo tiene una relación algo más directa,
    siendo, a su vez, de carácter más general. Incluyendo términos de gestión,
    planificación y metodología, lo cual no choca con lo que se esperaria si se
    hablara de economía y trabajo.
    \item \textbf{Grupo verde}: Este es de todos, el grupo más grande,
    conteniendo como término principal \textit{effect}. La alta presencia de
    este término es difícil de interpretar, pudiendo hacer referencia a estudios
    que analizan el efecto de medidas o investigaciones. Si se analizan las
    conexiones se puede ver que las mayorías son entre este grupo y el amarillo,
    pudiendo esto indicar un alto número de trabajos investigando el efecto de
    avances o políticas en el ámbito energético. Adicionalmente la presencia de
    términos como \textit{income} o \textit{gdp} puede indicar una mayor
    relación entre este grupo y el objetivo 8. De todas formas la presencia de
    estos términos en menor medida y relacionados con otros altamente
    especializados, como son los del grupo amarillo pude relevar al objetivo 8 a
    un segundo plano en cuanto a la clasificación.
    \item \textbf{Grupo morado}: Este último y menor grupo se encuentra en una
    posición más central en el diagrama indicando su presencia más general. Este
    está altamente relacionado con los recursos naturales y la comida. Términos
    que, como ya se ha visto con otros grupos, no extrañaría verlos de la mano
    de otros hablando de crecimiento económico.
\end{itemize}


\begin{figure}[H]
    \centering
    \includegraphics[scale=0.85]{Vos\_Sdg8}
    \captionsetup{justification=centering}
    \caption{Mapa de relaciones del objetivo 8}
    \label{Mapa de relaciones del objetivo 8}
\end{figure}

En general este diagrama, \cref{Mapa de relaciones del objetivo 8}, confirma que la
taxonomía de este conjunto está altamente dividida, de hecho se identifican
términos relacionados con más objetivos, como los son el 2 (relacionado con la
alimentación), el 7 (relacionado con la energía), el 3 (relacionado con la
salud) e incluso el 17 (relacionado con la gestión y las regulaciones). Es por
esto por lo que se decidió realizar un análisis más profundo.

Debido a la heterogeneidad identificada en los términos, se
decidió analizar el numero de clasificaciones totales generadas por el modelo 23
sobre el conjunto de datos del objetivo 8. Métricas representadas en la tabla,
\cref{table:Clasificaciones asbtracts ODS8 - modelo 23}. Como se puede apreciar
en esta, el número de clasificaciones está muy repartido
entre varios objetivos, destacando los objetivos, 3, 7, 8, 9, 12 y 17,
obteniendo más de 100 clasificaciones. Este patrón encaja casi a la perfección
con las conclusiones extraídas del mapa generado por VosViewer, \cref{Mapa de
relaciones del objetivo 8}, en el que se identificaron todos estos objetivos, a
excepción del 9 y el 12, en la taxonomía. Destacar también la presencia, aunque
en menor medida, del objetivo 2, obteniendo 93 clasificaciones, lo cual coincide
con la presencia del quinto grupo identificado en la \cref{Mapa de relaciones
del objetivo 8}.

\begin{table}[H]
    \begin{tabular}{| c | c |}
        \hline
        Objetivo & Clasificaciones \\
        \hline \hline
        ODS1   & 45  \\ \hline
        ODS2   & 93  \\ \hline
        ODS3   & 126 \\ \hline
        ODS4   & 18  \\ \hline
        ODS5   & 23  \\ \hline
        ODS6   & 33  \\ \hline
        ODS7   & 173 \\ \hline
        \textbf{ODS8}   & \textbf{112} \\ \hline
        ODS9   & 137 \\ \hline
        ODS10  & 21  \\ \hline
        ODS11  & 75  \\ \hline
        ODS12  & 121 \\ \hline
        ODS13  & 81  \\ \hline
        ODS14  & 19  \\ \hline
        ODS15  & 63  \\ \hline
        ODS16  & 12  \\ \hline
        ODS17  & 127 \\ \hline
    \end{tabular}
    \caption{Clasificaciones asbtracts ODS8 - modelo 23}
    \label{table:Clasificaciones asbtracts ODS8 - modelo 23}
\end{table}


\subsubsection{Análisis objetivo 9}
A continuación se muestra, en al ráfica, \cref{Mapa de relaciones del objetivo 9}, el
diagrama relacionado con el objetivo 9, este obtuvo, en la clasificaciones del
objetivo 23, \cref{table:Clasificaciones modelos sigmoid - datos extraidos de
internet/finales}, un resultado ligeramente superior al del objetivo 8, por lo
que se esperaría encontrar una taxonomía mas relacionada con el objetivo 9,
conteniendo términos relacionados con la industria, innovación e
infraestructura. 

Analizando la figura \cref{Mapa de relaciones del objetivo 9} se pueden identificar unos
términos principales como \textit{relationship}, \textit{literature},
\textit{affordable housing} y \textit{building} divididos en cuatro grupos:
\begin{itemize}
    \item \textbf{Grupo rojo}: este grupo incluye términos relacionados con las
    corporaciones y relaciones. De el diagrama no se puede extraer un tema
    concreto de este grupo. 
    \item \textbf{Grupo verde}: este grupo habla de temas como las cadenas de
    producción y en un primer plano, el termino ``literatura''. Esto resulta
    ciertamente extraño ya que es un término que, aunque pudiendo tener un papel
    central en cuanto a número de ocurrencias, no está directamente relacionado
    con ningún objetivo.
    \item \textbf{Grupo azul}: este es uno de los dos grupos más claros y con un
    tema principal más identificable, en este caso este tema es la viviendo, en
    concreto la vivienda asequible, la construcción y la infraestructura. Temas
    más directamente relacionados con el objetivo 9 aunque la se podría
    argumentar que la vivienda está más relacionada con el objetivo 11.
    \item \textbf{Grupo amarillo}: este último grupo es el más pequeño de todos,
    contando con términos sobre energía, construcciones y viviendas pero desde
    un puto de vista diferente al azul. Este también se podría identificar como
    relacionado con el objetivo 9 pero al igual que el azul también se podría
    argumentar que esta relación es más directa con otros objetivos como el 7
    o el 11.
\end{itemize}


\begin{figure}[H]
    \centering
    \includegraphics[scale=0.85]{Vos\_Sdg9}
    \captionsetup{justification=centering}
    \caption{Mapa de relaciones del objetivo 9}
    \label{Mapa de relaciones del objetivo 9}
\end{figure}

Como análisis final de este objetivo se analizaron las clasificaciones generadas
por el modelo 23, de igual manera que se hizo con el objetivo 8. Estás,
representadas en la tabla \cref{table:Clasificaciones asbtracts ODS9 - modelo
23}, muestran unos objetivos principales que son el 12 y el 17 pudiendo estar
relacionados con los grupos verde y rojo respectivamente ya que los términos
presentes en estos están más relacionados con estos dos objetivos que con el
resto. También destacar las clasificaciones asignadas a los objetivos 9 y 11,
esto coincide con lo analizado en los grupos amarillo y azul, en los cuales se
identificaron términos relacionados con estos objetivos. También mencionar que la
presencia de términos energéticos en el grupo amarillo se ve reflejada en las
clasificaciones del objetivo 7. 

\begin{table}[H]
    \begin{tabular}{| c | c |}
        \hline
        Objetivo & Clasificaciones \\
        \hline \hline
        ODS1   & 6   \\ \hline
        ODS2   & 46  \\ \hline
        ODS3   & 27  \\ \hline
        ODS4   & 0   \\ \hline
        ODS5   & 2   \\ \hline
        ODS6   & 5   \\ \hline
        ODS7   & 84  \\ \hline
        ODS8   & 27  \\ \hline
        \textbf{ODS9}   & \textbf{184} \\ \hline
        ODS10  & 2   \\ \hline
        ODS11  & 140 \\ \hline
        ODS12  & 302 \\ \hline
        ODS13  & 42  \\ \hline
        ODS14  & 4   \\ \hline
        ODS15  & 13  \\ \hline
        ODS16  & 5   \\ \hline
        ODS17  & 215 \\ \hline
    \end{tabular}
    \caption{Clasificaciones asbtracts ODS9 - modelo 23}
    \label{table:Clasificaciones asbtracts ODS9 - modelo 23}
\end{table}


\subsubsection{Análisis objetivo 10}
Este objetivo es el más importante de analizar ya que es el que menos
clasificaciones tiene asignadas por el modelo 23, un total de 31,
\cref{table:Clasificaciones modelos sigmoid - datos extraidos de
internet/finales}. Este objetivo tiene como tema principal la reducción de las
desigualdades por lo que se esperaria encontrar términos sociales. 

Como se puede apreciar en el diagrama, \cref{Mapa de relaciones del objetivo 10}, hay
multitud de términos relacionados, divididos en cuatro grupos aunque se
encuentran distribuidos de una manera más homogénea,  demostrando que no hay
tantos temas diferenciados. Estos 4 grupos principales son los siguientes:
\begin{itemize}
    \item \textbf{Grupo verde}: Este primer grupo es de los más extensos,
    incluyendo términos como \textit{China}, \textit{export} y \textit{tariff}
    indicando un tema muy diferente al objetivo 10, estando este más relacionado
    con las relaciones internacionales, tema principal del objetivo 17 e
    incluyendo términos como industria y agricultura, propios de los objetivos 8
    y 2 respectivamente.
    \item \textbf{Grupo rojo}: este grupo tiene una estructura un tanto caótica,
    estando poco aglomerado, las ocurrencias de los términos de este grupo son
    menores que las del resto y, al no tener un tema en concreto, no se espera
    que tenga un impacto muy relevante en las clasificaciones.
    \item \textbf{Grupo amarillo}: este es el segundo grupo poco aglomerado,
    estando en su mayoría entrelazado con el rojo. Presenta términos más
    relevantes y menos generales como lo son \textit{countries},
    \textit{dispute} y \textit{participation}. Estos términos tampoco están muy
    relacionados con el objetivo 10, haciendo referencia a temas más
    diplomáticos propios de otros objetivos, principalmente el 17. 
    \item \textbf{Grupo azul}: este grupo final tiene un elemento central
    \textit{access} relacionado con otros términos como \textit{protección} o
    \textit{intelectual propery}. El tema principal de este grupo no está tan
    claro ya que no hay ningún termino principal que lo aclare, si no uno
    mayoritario que conecta multitud de términos diferentes.
\end{itemize}

Como da a entender esta figura, la taxonomía presente en los datos del objetivo
10 no incluyen términos relacionados con igualdad ni con acciones sociales. Es
por esto por lo que los modelos no clasifican ninguno de los textos incluidos
en este conjunto como relacionados con el objetivo 10.



\begin{figure}[H]
    \centering
    \includegraphics[scale=0.85]{Vos\_Sdg10}
    \captionsetup{justification=centering}
    \caption{Mapa de relaciones del objetivo 10}
    \label{Mapa de relaciones del objetivo 10}
\end{figure}

Como análisis adicional y para confirmar el análisis de los términos revelados
en la \cref{Mapa de relaciones del objetivo 10}, se incluyen en la tabla,
\cref{table:Clasificaciones asbtracts ODS10 - modelo 23}, las clasificaciones
totales generadas por el modelo 23 sobre los datos del objetivo 10. En esta se
aprecia como solo 31 textos son clasificados como relacionados con el objetivo
10, mientras que al rededor de 100 son asignados a los modelos 2 y 8 estando
estos relacionados en mayor o menor medida con los términos presentes en el
grupo verde. Finalmente 564 son clasificados como relacionados con el objetivo
17, esto concuerda con gran parte de la taxonomía identificada. 

\begin{table}[H]
    \begin{tabular}{| c | c |}
        \hline
        Objetivo & Clasificaciones \\
        \hline \hline
        ODS1   & 10  \\ \hline
        ODS2   & 115 \\ \hline
        ODS3   & 81  \\ \hline
        ODS4   & 2   \\ \hline
        ODS5   & 1   \\ \hline
        ODS6   & 3   \\ \hline
        ODS7   & 4   \\ \hline
        ODS8   & 111 \\ \hline
        ODS9   & 83  \\ \hline
        \textbf{ODS10}  & \textbf{31}  \\ \hline
        ODS11  & 1   \\ \hline
        ODS12  & 21  \\ \hline
        ODS13  & 18  \\ \hline
        ODS14  & 5   \\ \hline
        ODS15  & 10  \\ \hline
        ODS16  & 12  \\ \hline
        ODS17  & 564 \\ \hline
    \end{tabular}
    \caption{Clasificaciones asbtracts ODS10 - modelo 23}
    \label{table:Clasificaciones asbtracts ODS10 - modelo 23}
\end{table}


\subsubsection{Análisis objetivo 11}
A continuación se analizarán los datos asignados con el objetivo 11, este es un
objetivo curioso ya que, como se aprecia en la tabla \cref{table:Clasificaciones
modelos sigmoid - datos extraidos de internet/finales}, el modelo 23 asigna más
de 200 textos a este objetivo mientras que el resto no asigna más que 8 en el
mejor de las casos. Haciendo referencia a la misma tabla, la cual contiene los
mejores modelos conseguidos. 

Analizando la figura \cref{Mapa de relaciones del objetivo 11} se puede observar una
taxonomía con tres términos principales \textit{digestion}, \textit{scenario} y
\textit{municipal solid waste management}, estas, a priori, si están ciertamente
relacionadas con el objetivo 11, igual no de una forma común pero es indudable
que una buena gestión de los residuos es una tarea esencial si se desea
conseguir ciudades y comunidades sostenibles, tema principal del objetivo 11.
Adicionalmente se identifican tres grupos principales y un cuarto minoritario:
\begin{itemize}
    \item \textbf{Grupo verde}: Este grupo contiene términos relacionados con
    la recolección, transporte y gestión de residuos. Términos relacionados
    principalmente con el objetivo 12 pero, en menor medida, relacionados con el 11.
    \item \textbf{Grupo azul}: La taxonomía presente en este grupo está definida
    por términos como sistema gestor de residuos, reciclaje y calentamiento
    global. Este también está relacionado con el objetivo 12 pero de igual
    manera relacionado con el 11.
    \item \textbf{Grupo rojo}: Este es e último grupo principal, en el se
    definen términos como digestión, compost, salud, comida y biomasa. términos
    altamente relacionados con el resto de grupos, principalmente la gestión de
    residuos, y al igual que el resto relacionados con el objetivo 12 y en
    menor medida con el 11. 
    \item \textbf{Grupo amarillo}: Este último grupo está presente en mucha
    menor medida que el resto, analizarlo con al resolución generada no es
    posible ya que no se pueden identificar correctamente los términos.
\end{itemize}
 
\begin{figure}[H]
    \centering
    \includegraphics[scale=0.85]{Vos\_Sdg11}
    \captionsetup{justification=centering}
    \caption{Mapa de relaciones del objetivo 11}
    \label{Mapa de relaciones del objetivo 11}
\end{figure}

Para confirmar las conclusiones extraídas de la \cref{Mapa de relaciones
del objetivo 11}, se han incluido las clasificaciones totales del modelo 23
sobre todos los textos del objetivo 11, \cref{table:Clasificaciones asbtracts
ODS11 - modelo 23}. En esta tabla se aprecia como casi 800 textos han sido
clasificados como relacionados con el objetivo 12, mientras que 202 con el 11.
Esto aunque dentro de lo que se esperaba, no se esperaba una diferencia tan
grande. Destacar el numero de clasificaciones asignadas al objetivo 7, al
rededor de 160, en este caso, en la taxonomía no se puede apreciar ningún
termino relacionado con la energía, posiblemente sea fruto de múltiples
publicaciones sobre generación de energía por medio de los residuos, pero esto
es únicamente especulación.

\begin{table}[H]
    \begin{tabular}{| c | c |}
        \hline
        Objetivo & Clasificaciones \\
        \hline \hline
        ODS1   & 0   \\ \hline
        ODS2   & 27  \\ \hline
        ODS3   & 48  \\ \hline
        ODS4   & 2   \\ \hline
        ODS5   & 0   \\ \hline
        ODS6   & 28  \\ \hline
        ODS7   & 164 \\ \hline
        ODS8   & 5   \\ \hline
        ODS9   & 10  \\ \hline
        ODS10  & 1   \\ \hline
        \textbf{ODS11}  & \textbf{202} \\ \hline
        ODS12  & 782 \\ \hline
        ODS13  & 35  \\ \hline
        ODS14  & 5   \\ \hline
        ODS15  & 9   \\ \hline
        ODS16  & 0   \\ \hline
        ODS17  & 21  \\ \hline
    \end{tabular}
    \caption{Clasificaciones asbtracts ODS11 - modelo 23}
    \label{table:Clasificaciones asbtracts ODS11 - modelo 23}
\end{table}


\subsubsection{Análisis objetivo 12}
El caso de este objetivo es el contrario al anterior, el 11. En este el modelo
23 asigna este objetivo a menos textos que el resto de modelos. En este caso la
gráfica, \cref{Mapa de relaciones del objetivo 12}, indica que la taxonomía de los datos de
este modelo está altamente relacionada con temas de investigación, siendo la mayoría de los
términos principales relacionados con este tema. A pesar de estar divididos en
cuatro grupos, los temas de todos son similares. 

De esta taxonomía no se puede sacar conclusiones claras, no teniendo términos
significativos relacionados con ningún objetivo. La única explicación posible es
que las publicaciones extraídas sean de temas muy diversos y no haya ningún
tema principal entre ellas.

\begin{figure}[H]
    \centering
    \includegraphics[scale=0.85]{Vos\_Sdg12}
    \captionsetup{justification=centering}
    \caption{Mapa de relaciones del objetivo 12}
    \label{Mapa de relaciones del objetivo 12}
\end{figure}

En la tabla \cref{table:Clasificaciones asbtracts ODS12 - modelo 23} se aprecian
todas las clasificaciones asignadas a los datos del objetivo 12. En estas se
aprecia que se clasifican sobre todo los objetivos 8, 9, 12 y mayoritariamente
el 17. En este caso no se identifica ninguna relación significativa entre la
taxonomía extraída de los datos y las clasificaciones ya que esta primera es
altamente ambigua y no aporta casi información relacionada con los objetivos.
\begin{table}[H]
    \begin{tabular}{| c | c |}
        \hline
        Objetivo & Clasificaciones \\
        \hline \hline
        ODS1   & 2   \\ \hline
        ODS2   & 9   \\ \hline
        ODS3   & 8   \\ \hline
        ODS4   & 2   \\ \hline
        ODS5   & 11  \\ \hline
        ODS6   & 2   \\ \hline
        ODS7   & 14  \\ \hline
        ODS8   & 224 \\ \hline
        ODS9   & 189 \\ \hline
        ODS10  & 3   \\ \hline
        ODS11  & 40  \\ \hline
        \textbf{ODS12}  & \textbf{125} \\ \hline
        ODS13  & 4   \\ \hline
        ODS14  & 0   \\ \hline
        ODS15  & 8   \\ \hline
        ODS16  & 2   \\ \hline
        ODS17  & 311 \\ \hline
    \end{tabular}
    \caption{Clasificaciones asbtracts ODS12 - modelo 23}
    \label{table:Clasificaciones asbtracts ODS12 - modelo 23}
\end{table}


\subsubsection{Análisis objetivo 13}
Este objetivo consigue un número de clasificaciones mayor que los abordados
hasta ahora. Es por esto que no se ha considerado necesario realizar un análisis
tan en profundidad. Analizando la \cref{Mapa de relaciones del objetivo 13} se
aprecia que uno de los 3 grupos, y a su vez el más grande en cuanto a magnitud,
está directamente relacionado con este objetivo, incluyendo términos como cambio
climático e impacto. Adicionalmente la existencia de otros dos grupos
minoritarios con temas diferentes como lo son la vegetación, o los depósitos y
sedimentación explica que se hayan clasificado un número tan relativamente bajo
de textos como relacionados con el objetivo 13.


\begin{figure}[H]
    \centering
    \includegraphics[scale=0.85]{Vos\_Sdg13}
    \captionsetup{justification=centering}
    \caption{Mapa de relaciones del objetivo 13}
    \label{Mapa de relaciones del objetivo 13}
\end{figure}

\subsubsection{Análisis objetivo 14}
Estees el último de los objetivos que tiene una clasificación baja, 118 textos
asignados por el modelo 23, \cref{table:Clasificaciones modelos sigmoid - datos
extraidos de internet/finales}. Esto contrasta con lo visto en la
\cref{Mapa de relaciones del objetivo 14} ya que en esta se aprecia como la mayoría de
los términos están relacionados con el agua, aún divididos en cuatro grupos,
todos ellos están relacionados con este tema.

\begin{figure}[H]
    \centering
    \includegraphics[scale=0.85]{Vos\_Sdg14}
    \captionsetup{justification=centering}
    \caption{Mapa de relaciones del objetivo 14}
    \label{Mapa de relaciones del objetivo 14}
\end{figure}

Como explicación se han incluido, \cref{table:Clasificaciones asbtracts ODS14 -
modelo 23}, las clasificaciones de todos los textos relacionados con el objetivo
11. Es esta tabla se aprecia como la mayoría de los textos han sido
clasificados como relacionados con el objetivo 6, aquel que habla sobre el la
calidad y saneamiento del agua. Viendo esto y analizando de nuevo la
\cref{Mapa de relaciones del objetivo 14} se puede entender el porque de este numero de
clasificaciones tan bajo, ya que la taxonomía extraída no hace referencia a los
ecosistemas ni ningún tipo de vida acuática por lo que está mas relacionado con
el objetivo 6 que con el 11, estando este último relacionado con los ecosistemas
y vida submarina.

\begin{table}[H]
    \begin{tabular}{| c | c |}
        \hline
        Objetivo & Clasificaciones \\
        \hline \hline
        ODS1   & 1   \\ \hline
        ODS2   & 35  \\ \hline
        ODS3   & 64  \\ \hline
        ODS4   & 0   \\ \hline
        ODS5   & 0   \\ \hline
        ODS6   & 589 \\ \hline
        ODS7   & 5   \\ \hline
        ODS8   & 0   \\ \hline
        ODS9   & 9   \\ \hline
        ODS10  & 1   \\ \hline
        \textbf{ODS11}  & \textbf{43}  \\ \hline
        ODS12  & 58  \\ \hline
        ODS13  & 97  \\ \hline
        ODS14  & 118 \\ \hline
        ODS15  & 65  \\ \hline
        ODS16  & 1   \\ \hline
        ODS17  & 4   \\ \hline
    \end{tabular}
    \caption{Clasificaciones asbtracts ODS14 - modelo 23}
    \label{table:Clasificaciones asbtracts ODS14 - modelo 23}
\end{table}


\subsubsection{Análisis objetivo 15}
Este objetivo consigue un número más alto de clasificaciones, 411,
\cref{table:Clasificaciones modelos sigmoid - datos extraidos de
internet/finales}. Esta, aunque alta, es más baja de lo que sería ideal. Esto
como se aprecia en la \cref{Mapa de relaciones del objetivo 15}, puede ser debido
a la división de la taxonomía en dos grandes grupos. Además esto concuerda con
la cantidad de textos asignados, ya que estos son aproximadamente la mitad del
total. Unos de los dos grupos, el verde, habla de especies y cambio climático,
temas que se podrían considerar como relacionados al objetivo 15. Adicionalmente
el otro grupo, el rojo, contiene términos como sistema complejo y adaptativo y
sostenibilidad, los cuales se pueden considerara más genéricos.


\begin{figure}[H]
    \centering
    \includegraphics[scale=0.85]{Vos\_Sdg15}
    \captionsetup{justification=centering}
    \caption{Mapa de relaciones del objetivo 15}
    \label{Mapa de relaciones del objetivo 15}
\end{figure}


\subsubsection{Análisis objetivo 16}
Este objetivo obtiene, al igual que el 15, unos resultados positivos, más aún si
se tiene en cuenta que el número total de textos es menor, pero el número de
clasificaciones es similar, \cref{table:Clasificaciones modelos sigmoid - datos
extraidos de internet/finales}. Esto está reforzado por la \cref{Mapa de
relaciones del objetivo 16}, en la cual se aprecia una taxonomía altamente
relaciona con el objetivo 16, cubriendo temas como los conflictos armados,
guerras civiles, entre otros. Esto cuadra correctamente con los resultados
obtenidos.

\begin{figure}[H]
    \centering
    \includegraphics[scale=0.85]{Vos\_Sdg16}
    \captionsetup{justification=centering}
    \caption{Mapa de relaciones del objetivo 16}
    \label{Mapa de relaciones del objetivo 16}
\end{figure}

\subsubsection{Análisis objetivo 17}
Este último objetivo, he obtenido unas métricas razonables pero aún así
relativamente bajas, \cref{table:Clasificaciones modelos sigmoid - datos
extraidos de internet/finales}. Esto se ve reflejado en la \cref{Mapa de
relaciones del objetivo 17}, presentando este una taxonomía altamente
diversificada, tocando temas de todos los tipos como lo son la industria, el
comercio, la pobreza y de todos ellos, el único realmente relacionado con el
objetivo 17 es la colaboración, presente en un segundo plano en el grupo rojo de
la figura.
\begin{figure}[H]
    \centering
    \includegraphics[scale=0.85]{Vos\_Sdg17}
    \captionsetup{justification=centering}
    \caption{Mapa de relaciones del objetivo 17}
    \label{Mapa de relaciones del objetivo 17}
\end{figure}

Para confirmar la diversidad en la taxonomía de los textos de este objetivo, se
han incluido, en la tabla, \cref{table:Clasificaciones asbtracts ODS17 - modelo
23}, las clasificaciones del modelo 23 de todos los textos pertenecientes al
conjunto de datos de este objetivo. En esta se ve como hay multitud de objetivos
con en torno a 100-150 textos asignados y de todos el que más tiene es el 17
contando con cerca de 300. Esto concuerda correctamente con lo visto en la
taxonomía, \cref{Mapa de relaciones del objetivo 17}, siendo esta bastante variada.

\begin{table}[H]
    \begin{tabular}{| c | c |}
        \hline
        Objetivo & Clasificaciones \\
        \hline \hline
        ODS1   & 43  \\ \hline
        ODS2   & 95  \\ \hline
        ODS3   & 159 \\ \hline
        ODS4   & 21  \\ \hline
        ODS5   & 9   \\ \hline
        ODS6   & 47  \\ \hline
        ODS7   & 93  \\ \hline
        ODS8   & 68  \\ \hline
        ODS9   & 125 \\ \hline
        ODS10  & 16  \\ \hline
        ODS11  & 65  \\ \hline
        ODS12  & 109 \\ \hline
        ODS13  & 129 \\ \hline
        ODS14  & 23  \\ \hline
        ODS15  & 59  \\ \hline
        ODS16  & 5   \\ \hline
        ODS17  & 298 \\ \hline
    \end{tabular}
    \caption{Clasificaciones asbtracts ODS17 - modelo 23}
    \label{table:Clasificaciones asbtracts ODS17 - modelo 23}
\end{table}

\subsection{Conclusiones parciales}
Gracias a estas pruebas podemos destacar el rendimiento del modelo 23,
obteniendo este unas clasificaciones, en la mayoría de los casos correctas y  en
las que no, un posterior análisis de los datos usados demostró que realmente el
modelo funcionaba según lo esperado pero los conjuntos no contenían datos
relacionados con el objetivo esperado. Este último análisis también ha ayudado a
determinar que resultados excepcionalmente buenos, como el del modelo 21,
\cref{table:Clasificaciones modelos sigmoid - datos aumentados}, realmente eran
demasiado buenos como para ser verdad, analizando únicamente la tabla puede
parecer que este modelo destaca por su rendimiento excepcional, obteniendo una
métricas muy superiores al resto, pero si se analizan las taxonomías de los
conjunto de datos , \crefrange{Mapa de relaciones del objetivo 8}{Mapa de
relaciones del objetivo 17}, se observa que realmente un comportamiento correcto
mostraría menos textos clasificados de manera correcta.

\section{Conclusiones generales}
Las pruebas realizadas pueden considerarse, en gran medida, como
satisfactorias, alojando luz sobre el funcionamiento y razonamiento detrás de
cada modelo. Han ayudado a determinar si las métricas obtenidas por los modelos
son correctas o no y finalmente ayudando a determinar cual de todos los modelos
presenta un comportamiento superior, como es el caso del modelo 23. 

Se puede argumentar que tal nivel de pruebas y de análisis puede llegar a ser
excesivo pero si el objetivo finales realizar un análisis cuantitativo, como es
el caso, la validez de los datos finales presentados depende directamente de la
validación realizada y, como es obvio de los resultados de la misma, y de que
esta sea suficientemente extensa y detallada. 
