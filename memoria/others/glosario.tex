%%%%%%%%%%%%%%%%%%%%%%%%%%%%%%%%%%%%%%%%%%%%%%%%%%%%%%%%%%%%%%%%%%%%%%%%%%%
%% Entradas para glosario (recomendación: añadir "g" tras etiqueta)
%%%%%%%%%%%%%%%%%%%%%%%%%%%%%%%%%%%%%%%%%%%%%%%%%%%%%%%%%%%%%%%%%%%%%%%%%%%
\newglossaryentry{MFAg}{
    name={ Medidas Fiscales y Administrativas},
    description={La Ley 3/2008, del 23 de diciembre, sobre Medidas Fiscales y Administrativas es una normativa española que contempla un conjunto de disposiciones en el ámbito fiscal y administrativo. Aunque su nombre sugiere una amplia cobertura, es necesario consultar el texto específico de la ley para determinar los aspectos concretos que regula, ya que las leyes de "Medidas Fiscales y Administrativas" suelen ser normativas de carácter anual o periódico que ajustan, modifican o introducen disposiciones en diferentes ámbitos relacionados con la administración y la fiscalidad.}
}

\newglossaryentry{ODSg}{ 
    name={Objetivos de Desarrollo Sostenible},
    description={Los ODS, o Objetivos de Desarrollo Sostenible, son un conjunto
    de 17 metas globales establecidas por las Naciones Unidas en 2015 como parte
    de la Agenda 2030 para abordar problemas mundiales urgentes, como la
    pobreza, el hambre, la igualdad de género, la educación, la salud, el medio
    ambiente y la paz. Estos objetivos tienen como objetivo mejorar la calidad
    de vida de las personas y proteger el planeta, promoviendo un desarrollo
    sostenible a nivel económico, social y ambiental.} }

\newglossaryentry{ONUg}{ 
    name={Organización de las Naciones Unidas},
    description={La ONU, o Organización de las Naciones Unidas, es una
    organización internacional establecida en 1945 para promover la cooperación
    y la paz mundial. Su objetivo principal es facilitar la diplomacia y la
    resolución pacífica de conflictos entre países, así como abordar problemas
    globales como los derechos humanos, el desarrollo sostenible y la seguridad
    internacional. La ONU consta de 193 Estados miembros y desempeña un papel
    fundamental en la promoción de la cooperación internacional y la protección
    de los derechos y el bienestar de las personas en todo el mundo.} 
}

\newglossaryentry{IAg}{ 
    name={Inteligencia Artificial},
    description={La IA, o Inteligencia Artificial, es un campo de la informática
    que se enfoca en desarrollar sistemas y programas que pueden realizar tareas
    que normalmente requieren inteligencia humana, como el aprendizaje, la toma
    de decisiones, el reconocimiento de patrones y la resolución de problemas.
    Utiliza algoritmos y datos para simular procesos cognitivos y mejorar la
    automatización y el rendimiento en una amplia variedad de aplicaciones,
    desde asistentes virtuales hasta vehículos autónomos.} }
    
\newglossaryentry{EDAg}{ 
    name={Easy Data Augmentation},
    description={Easy Data Augmentation (EDA) es una técnica utilizada en el procesamiento de lenguaje natural (NLP) y el aprendizaje automático para aumentar la cantidad de datos de entrenamiento. EDA consiste en aplicar transformaciones simples y controladas a los datos de entrenamiento existentes para generar nuevas instancias de datos. Estas transformaciones pueden incluir sinónimos, eliminación de palabras, inserción de palabras, y permutación de palabras, entre otros. EDA es útil para mejorar el rendimiento de los modelos de NLP al proporcionar más variabilidad en los datos de entrenamiento, lo que a menudo resulta en un mejor rendimiento en tareas como la clasificación de texto y la traducción automática.} 
}
    
\newglossaryentry{BERTg}{ 
    name={Bidirectional Encoder Representations from Transformers},
    description={BERT se refiere a Bidirectional Encoder Representations from Transformers. Es un modelo de lenguaje desarrollado por Google en 2018 para tareas de procesamiento del lenguaje natural (NLP, por sus siglas en inglés). Lo que distingue a BERT de otros modelos es su capacidad para entrenar representaciones de palabras basadas en su contexto en ambas direcciones (izquierda y derecha) de una palabra en una frase, lo que lo hace bidireccional. Esto permite a BERT tener un desempeño superior en muchas tareas de NLP, desde la comprensión de texto hasta la respuesta a preguntas.} }
    
\newglossaryentry{WebOfScienceg}{ 
    name={Web of Science},
    description={Web of Science es una base de datos de investigación en línea que proporciona citas y referencias de literatura académica en diversas disciplinas. Es una herramienta valiosa para los académicos y investigadores que buscan literatura relevante en su campo, y es ampliamente utilizada para la búsqueda de artículos científicos, procedimientos de conferencias y otras publicaciones. También es una herramienta comúnmente utilizada para análisis de citas y para determinar el impacto de las publicaciones y autores en la academia.} }

\newglossaryentry{HTTPg}{
    name={Protocolo de Transferencia de HiperTexto},
    description={HTTP es el protocolo utilizado en la web para solicitar y enviar información entre navegadores y servidores}
}


\newglossaryentry{NLPg}{
    name={Procesamiento del Lenguaje Natural},
    description={ El Procesamiento del Lenguaje Natural (Natural Language Processing en inglés) es una rama de la inteligencia artificial que se centra en la interacción entre las computadoras y el lenguaje humano, permitiendo a las máquinas entender, interpretar y generar lenguaje de manera similar a como lo hacen los humanos. NLP es esencial en aplicaciones como la traducción automática, el análisis de sentimiento, y los chatbots, entre otros.}
}
\newglossaryentry{HTMLg}{
    name={Lenguaje de Marcado de Hipertexto},
     description={Lenguaje de Marcado de Hipertexto (HTML) es el lenguaje estándar utilizado para crear y diseñar páginas web. Define la estructura y el contenido, como texto, imágenes y otros elementos, y puede ser complementado con CSS y JavaScript para añadir estilos y funcionalidad, respectivamente}
}
\newglossaryentry{cvsg}{
    name={Comma Separated Values},
    description={CSV, o Comma Separated Values (Valores Separados por Comas), es un formato de archivo utilizado para almacenar datos tabulares en forma de texto plano. En un archivo CSV, los datos se organizan en filas y columnas, donde cada fila representa un registro y los valores de cada columna están separados por comas u otro delimitador. Los archivos CSV son ampliamente utilizados en aplicaciones de hojas de cálculo, bases de datos y sistemas de intercambio de datos debido a su simplicidad y facilidad de lectura y escritura. Este formato es especialmente útil para importar y exportar datos entre diferentes programas y plataformas.}
}

\newglossaryentry{CBOWg}{
    name={Continuous Bag of Words},
     description={"CBOW" se refiere a "Continuous Bag of Words". Es un modelo utilizado en el entrenamiento de word embeddings, específicamente en el marco Word2Vec. CBOW predice palabras objetivo (por ejemplo, palabras centrales) a partir de sus palabras de contexto circundantes en una ventana determinada, a diferencia del modelo Skip-Gram de Word2Vec que hace lo contrario: predice palabras de contexto a partir de una palabra objetivo. Ambos métodos son técnicas para generar representaciones vectoriales densas de palabras a partir de grandes corpus de texto.}
}
\newglossaryentry{SGg}{
    name={Continuous Skip-gram },
     description= {El modelo "Skip-gram", parte del marco Word2Vec, tiene como objetivo predecir palabras circundantes (contexto) a partir de una palabra dada. Es una técnica para generar representaciones vectoriales (embeddings) de palabras de manera que palabras con contextos similares queden cercanas en un espacio vectorial.
}}
\newglossaryentry{JOSSg}{
    name={Journal of Open Source Software },
     description= {El "Journal of Open Source Software" (JOSS) es una revista académica con revisión por pares que se centra en la publicación de software de investigación de código abierto. Los artículos publicados en JOSS son breves y destinados principalmente a presentar y describir el software, su utilidad y relevancia en la investigación científica. La principal ventaja de publicar en JOSS es que promueve la práctica de hacer que el software científico sea de código abierto y revisable, lo que a su vez mejora la reproducibilidad y la transparencia en la investigación.}}

\newglossaryentry{UNEPg}{
    name={Programa de las Naciones Unidas para el Medio Ambiente},
    description={La principal autoridad mundial en asuntos ambientales, establecida por las Naciones Unidas en 1972. Su objetivo es promover la conservación y el uso sostenible de los recursos naturales, así como abordar los desafíos ambientales globales. Trabaja en colaboración con gobiernos, organizaciones y la sociedad civil para proteger el medio ambiente y fomentar un desarrollo sostenible en todo el mundo.}
}

\newglossaryentry{APIg}{
    name={Application Programming Interface},
    description={Una API, o Interfaz de Programación de Aplicaciones, es un conjunto de reglas y protocolos que permite que diferentes software se comuniquen entre sí. Proporciona un conjunto de funciones y procedimientos que los desarrolladores pueden utilizar para interactuar con una aplicación o sistema informático específico. Las APIs son fundamentales para la integración de servicios y la creación de aplicaciones de software que aprovechan las funcionalidades de otros programas o plataformas.}
}

\newglossaryentry{PDFg}{
    name={Portable Document Format},
    description={El formato PDF, o Portable Document Format, es un formato de archivo diseñado para presentar documentos de manera independiente del software, el hardware y el sistema operativo utilizados para crearlos. Los archivos PDF pueden contener texto, imágenes, gráficos y otros elementos y mantienen su formato y diseño original cuando se visualizan en diferentes dispositivos y plataformas. Son ampliamente utilizados para compartir documentos, formularios, libros electrónicos y otros tipos de contenido digital.}
}

\newglossaryentry{UN DESAg}{
    name={Departamento de Asuntos Económicos y Sociales de las Naciones Unidas},
    description={El Departamento de Asuntos Económicos y Sociales de las Naciones Unidas (UN DESA) es una entidad de las Naciones Unidas que se encarga de promover el desarrollo sostenible y la cooperación internacional en asuntos económicos y sociales. El Departamento desempeña un papel clave en la investigación y análisis de cuestiones económicas, sociales y ambientales a nivel global. También brinda apoyo técnico y asesoramiento a los estados miembros de las Naciones Unidas en áreas como el desarrollo económico, social y ambiental.}
}

\newglossaryentry{UPMg}{
    name={Universidad Politécnica de Madrid},
    description={La Universidad Politécnica de Madrid (UPM) es una institución de educación superior ubicada en Madrid, España. Es una de las principales universidades técnicas de España y ofrece una amplia gama de programas académicos en áreas como la ingeniería, la arquitectura, la informática y las ciencias aplicadas. La UPM se dedica a la investigación y la formación en tecnología y ciencia, y desempeña un papel importante en la promoción de la innovación y el desarrollo tecnológico en el país.}
}

\newglossaryentry{GloVeg}{
    name={Global Vectors for Word Representation},
    description={Global Vectors for Word Representation (GloVe) es un algoritmo de aprendizaje no supervisado que se utiliza para representar palabras en un espacio vectorial continuo. GloVe se utiliza comúnmente en el procesamiento del lenguaje natural (NLP) y la recuperación de información para capturar las relaciones semánticas entre palabras. Este algoritmo mapea palabras en vectores numéricos de manera que palabras con significados similares están ubicadas cerca unas de otras en el espacio vectorial. GloVe se utiliza ampliamente en tareas de NLP, como la traducción automática, el análisis de sentimientos y la recuperación de información.}
}

\newglossaryentry{USEg}{
    name={Universal Sentence Encoders},
    description={Universal Sentence Encoders (USE) es un conjunto de modelos de procesamiento de lenguaje natural desarrollados por Google. Estos modelos están diseñados para convertir oraciones o fragmentos de texto en vectores numéricos, lo que facilita la comparación y el análisis de similitud semántica entre oraciones en tareas de procesamiento de lenguaje natural. Los USE se utilizan en una variedad de aplicaciones, como la búsqueda semántica, la recuperación de información y la clasificación de texto. Estos modelos capturan representaciones de alta calidad para oraciones en varios idiomas y se pueden utilizar en una amplia gama de aplicaciones de NLP.}
}

\newglossaryentry{TF-IDFg}{
    name={Term Frequency-Inverse Document Frequency},
    description={TF-IDF, o Term Frequency-Inverse Document Frequency (Frecuencia de Término-Frecuencia Inversa de Documento), es una técnica utilizada en el procesamiento de texto y la recuperación de información. Se utiliza para evaluar la importancia de una palabra o término en un documento dentro de un corpus de documentos. TF-IDF combina dos métricas: la frecuencia de término (TF), que mide la frecuencia de una palabra en un documento específico, y la frecuencia inversa de documento (IDF), que mide la importancia de un término en todo el corpus. La puntuación TF-IDF se utiliza comúnmente para la representación de documentos y la recuperación de información, permitiendo identificar palabras clave o términos relevantes en un contexto dado.}
}

\newglossaryentry{LDAg}{
    name={Latent Dirichlet Allocation},
    description={Latent Dirichlet Allocation (LDA) es un modelo generativo utilizado en el procesamiento de lenguaje natural y el análisis de texto. LDA es una técnica de aprendizaje no supervisado que se utiliza para descubrir temas latentes en un conjunto de documentos. El modelo asume que cada documento está compuesto por una mezcla de temas, y cada tema está compuesto por una mezcla de palabras. LDA se utiliza comúnmente en tareas como la agrupación de documentos, la clasificación de texto y la extracción de temas. Es una herramienta valiosa para analizar grandes colecciones de texto y descubrir patrones y estructuras subyacentes en los datos.}
}

\newglossaryentry{ASCIIg}{
    name={American Standard Code for Information Interchange},
    description={El American Standard Code for Information Interchange (ASCII) es un conjunto de caracteres y códigos de control utilizados para representar texto y controlar dispositivos en sistemas informáticos y de telecomunicaciones. ASCII utiliza números de 7 bits para representar caracteres alfabéticos, numéricos y especiales, lo que permite la comunicación de texto en una amplia variedad de dispositivos y plataformas informáticas. Es un estándar ampliamente utilizado en la codificación de texto y proporciona una forma común de representar caracteres en la mayoría de las computadoras y sistemas de comunicación.}
}

\newglossaryentry{UTF-8g}{
    name={Unicode Transformation Format 8-bit},
    description={UTF-8 es un estándar de codificación de caracteres que representa la mayoría de los caracteres del conjunto de caracteres Unicode utilizando secuencias de bytes de 8 bits. UTF-8 es un sistema de codificación ampliamente utilizado en la informática y la comunicación digital, ya que permite representar una amplia variedad de caracteres, incluyendo caracteres alfabéticos, numéricos, símbolos y caracteres especiales de diferentes idiomas y escrituras. Es compatible con ASCII y es el estándar de codificación predeterminado en muchas aplicaciones y sistemas operativos.}
}

\newglossaryentry{ADAMg}{
    name={Adaptive Moment Estimation},
    description={ADAM es un algoritmo de optimización utilizado en el aprendizaje automático y el entrenamiento de redes neuronales artificiales. Combina conceptos de otros algoritmos de optimización, como el descenso de gradiente estocástico (SGD) y RMSprop, para ajustar dinámicamente la tasa de aprendizaje durante el proceso de entrenamiento. Es conocido por su eficacia en la convergencia rápida y la adaptación a diferentes tasas de aprendizaje para diferentes parámetros del modelo. Es ampliamente utilizado en la optimización de redes neuronales profundas y otros modelos de aprendizaje automático.}
}

\newglossaryentry{IISDg}{
    name={International Institute for Sustainable Development},
    description={El International Institute for Sustainable Development (IISD) es una organización sin fines de lucro dedicada a la promoción del desarrollo sostenible a nivel global. El IISD se enfoca en la investigación, análisis y promoción de políticas y prácticas que contribuyan a un desarrollo económico, social y ambiental equitativo y sostenible. La organización trabaja en una variedad de áreas, incluyendo el cambio climático, la conservación de recursos naturales, el comercio sostenible y la gobernanza ambiental. El IISD colabora con gobiernos, empresas y organizaciones de la sociedad civil para abordar los desafíos globales relacionados con la sostenibilidad.}
}

\newglossaryentry{UNDPg}{ 
    name={United Nations Development Programme},
    description={El Programa de las Naciones Unidas para el Desarrollo (PNUD o UNDP en inglés) es la red global de desarrollo de las Naciones Unidas. Ayuda a los países a alcanzar los Objetivos de Desarrollo Sostenible (ODS) y a eliminar la pobreza en todas sus formas y dimensiones. El PNUD trabaja en áreas como la reducción de la desigualdad, la promoción de la igualdad de género, la construcción de capacidades y la sostenibilidad ambiental.} 
}

%%%%%%%%%%%%%%%%%%%%%%%%%%%%%%%%%%%%%%%%%%%%%%%%%%%%%%%%%%%%%%%%%%%%%%%%%%%
%% Entradas para acrónimos (recomendación: añadir "a" tras etiqueta)
%%%%%%%%%%%%%%%%%%%%%%%%%%%%%%%%%%%%%%%%%%%%%%%%%%%%%%%%%%%%%%%%%%%%%%%%%%%

\newglossaryentry{cvsa}{
    type = \acronymtype,
    name = {CSV},
    description = {Comma Separated Values},
    see = [Glosario:]{cvsg}
}
\newglossaryentry{BERTa}{
    type = \acronymtype,
    name = {BERT},
    description = {Bidirectional Encoder Representations from Transformers},
    see = [Glosario:]{BERTg}
}

\newglossaryentry{WebOfSciencea}{
    type = \acronymtype,
    name = {WoS},
    description = {Web of Science},
    see = [Glosario:]{WebOfScienceg}
}

\newglossaryentry{ODSa}{
    type = \acronymtype,
    name = {ODS},
    description = {Objetivos de Desarrollo Sostenible},
    see = [Glosario:]{ODSg}
}
\newglossaryentry{EDAa}{
    type = \acronymtype,
    name = {EDA},
    description = {Easy Data Augmentation},
    see = [Glosario:]{EDAg}
}

\newglossaryentry{ONUa}{
    type = \acronymtype,
    name = {ONU},
    description = {Organización de las Naciones Unidas},
    see = [Glosario:]{ONUg}
}

\newglossaryentry{IAa}{
    type = \acronymtype,
    name = {IA},
    description = {Inteligencia Artificial},
    see = [Glosario:]{IAg}
}
\newglossaryentry{httpa}{
    type = \acronymtype,
    name = {HTTP},
    description = {Protocolo de Transferencia de HiperTexto},
    see=[Glosario:]{HTTPg}
}
\newglossaryentry{SGa}{
    type = \acronymtype,
    name = {SG},
    description = {Continuous Skip-gram},
    see=[Glosario:]{SGg}
}
\newglossaryentry{NLPa}{
    type = \acronymtype,
    name = {NLP},
    description = {Procesamiento del Lenguaje Natural},
    see=[Glosario:]{NLPg}
}
\newglossaryentry{htmla}{
    type = \acronymtype,
    name = {HTML},
    description = {HTML},
    see=[Glosario:]{HTMLg}
}


\newglossaryentry{CBOWa}{
    type = \acronymtype,
    name = {CBOW},
    description = {Continuous Bag of Words},
    see=[Glosario:]{CBOWg}
}
\newglossaryentry{JOSSa}{
    type = \acronymtype,
    name = {JOSS},
    description = {Journal of Open Source Software},
    see=[Glosario:]{JOSSg}
}

\newglossaryentry{UNEPa}{
    type = \acronymtype,
    name = {UNEP},
    description = {Programa de las Naciones Unidas para el Medio Ambiente},
    see=[Glosario:]{UNEPg}
}

\newglossaryentry{APIa}{
    type = \acronymtype,
    name = {API},
    description = {Interfaz de Programación de Aplicaciones},
    see = [Glosario:]{APIg}
}

\newglossaryentry{PDFa}{
    type = \acronymtype,
    name = {PDF},
    description = {Portable Document Format},
    see = [Glosario:]{PDFg}
}

\newglossaryentry{UN DESAa}{
    type = \acronymtype,
    name = {UN DESA},
    description = {Departamento de Asuntos Económicos y Sociales de las Naciones Unidas},
    see = [Glosario:]{UN DESAg}
}

\newglossaryentry{UPMa}{
    type = \acronymtype,
    name = {UPM},
    description = {Universidad Politécnica de Madrid},
    see = [Glosario:]{UPMg}
}

\newglossaryentry{GloVea}{
    type = \acronymtype,
    name = {GloVe},
    description = {Global Vectors for Word Representation},
    see = [Glosario:]{GloVeg}
}

\newglossaryentry{USEa}{
    type = \acronymtype,
    name = {USE},
    description = {Universal Sentence Encoders},
    see = [Glosario:]{USEg}
}

\newglossaryentry{TF-IDFa}{
    type = \acronymtype,
    name = {TF-IDF},
    description = {Term Frequency-Inverse Document Frequency},
    see = [Glosario:]{TFIDFg}
}

\newglossaryentry{LDAa}{
    type = \acronymtype,
    name = {LDA},
    description = {Latent Dirichlet Allocation},
    see = [Glosario:]{LDAg}
}

\newglossaryentry{ASCIIa}{
    type = \acronymtype,
    name = {ASCII},
    description = {American Standard Code for Information Interchange},
    see = [Glosario:]{ASCIIg}
}

\newglossaryentry{UTF-8a}{
    type = \acronymtype,
    name = {UTF-8},
    description = {Unicode Transformation Format 8-bit},
    see = [Glosario:]{UTF-8g}
}

\newglossaryentry{ADAMa}{
    type = \acronymtype,
    name = {ADAM},
    description = {Adaptive Moment Estimation},
    see = [Glosario:]{ADAMg}
}

\newglossaryentry{IISDa}{
    type = \acronymtype,
    name = {IISD},
    description = {International Institute for Sustainable Development},
    see = [Glosario:]{IISDg}
}

\newglossaryentry{UNDPa}{
    type = \acronymtype,
    name = {UNDP},
    description = {Programa de las Naciones Unidas para el Desarrollo},
    see = [Glosario:]{UNDPg}
}